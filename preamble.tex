%%% Preamble

\usepackage{amsfonts}
\usepackage{textcomp}
\usepackage{amsmath,amssymb}
\usepackage{amsthm}
\usepackage{graphicx}
\usepackage{rotating}
%\usepackage{natbib}
\usepackage[T1]{fontenc}
\usepackage{times}
\usepackage{mathptmx} % rm & math
\usepackage[scaled=0.90]{helvet} % ss
\normalfont
\usepackage[scaled=0.95]{inconsolata}
\usepackage{textcomp,wasysym,comment,titlesec,color,booktabs} %colortbl,
\usepackage[notintoc]{nomencl}
%\usepackage[table]{xcolor}
\usepackage{appendix}
% \usepackage{multirow,subfig}
\usepackage{relsize,units}
\usepackage[multiple]{footmisc}
\usepackage{ulem}
\usepackage{longtable}
\usepackage{url}
\usepackage{float}
\usepackage{pdfpages}
\usepackage[bookmarks=true, colorlinks=true, citecolor=black,
urlcolor=black, linkcolor=black]{hyperref}
%\usepackage{ifthen}
\normalem

\graphicspath{{figures/}}

%\bibpunct{(}{)}{;}{a}{,}{,}

\newtheorem{theorem}{Theorem}
\newtheorem{corollary}[theorem]{Corollary}
\newtheorem{definition}[theorem]{Definition}
\newtheorem{lemma}[theorem]{Lemma}

\newcommand{\baselinestrech}{3}

\definecolor{myGrey}{rgb}{0.85,0.85,0.85}

\newcommand{\nt}{\noindent}
\newcommand{\cd}{\ensuremath{\mathrm{d}}}
\newcommand{\del}{\ensuremath{\partial}}
\newcommand{\Del}[2]{\ensuremath{\frac{\partial{#1}}{\partial{#2}}}}
\newcommand{\Delsq}[2]{\ensuremath{\frac{\partial^2{#1}}{\partial{#2}^2}}}

\newcommand{\bracI}[1]{\ensuremath{\left(#1\right)}}
\newcommand{\bracII}[1]{\ensuremath{\left\{#1\right\}}}
\newcommand{\bracIII}[1]{\ensuremath{\left[#1\right]}}
\newcommand{\bracIV}[1]{\ensuremath{\left\langle#1\right\rangle}}

\newcommand{\etal}{\emph{et al.}}
\newcommand{\ml}{\mathlarger}

\includecomment{figs}
%\excludecomment{figs}

\hyphenation{gno-mon-ly}
\titlespacing*{\section}{0in}{*0}{0.2in}

\makenomenclature
%\RequirePackage{ifthen}
%\renewcommand{\nomgroup}[1]{%
%\ifthenelse{\equal{#1}{A}}{\item[\textbf{Roman Symbols}]}{%
%\ifthenelse{\equal{#1}{G}}{\vspace{0.5cm}\item[\textbf{Greek Symbols}]}{%
%\ifthenelse{\equal{#1}{Z}}{\vspace{0.5cm}\item[\textbf{Abbreviations}]}{%
%\ifthenelse{\equal{#1}{S}}{\vspace{0.5cm}\item[\textbf{Subscripts/Superscripts}]}
%{}
%}% matches Subscripts
%}% matches Abbreviations
%}% matches Greek Symbols
%}% matches Roman Symbols

\hoffset        -0.00in
\voffset         0.0in

%My own added commands
\newcommand{\ee}{\end{equation}}
\newcommand{\be}{\begin{equation}}
\newcommand{\bi}{\begin{itemize}}
\newcommand{\ei}{\end{itemize}}
\newcommand{\bc}{\begin{center}}
\newcommand{\ec}{\end{center}}
\newcommand{\mc}{\multicolumn}

\newcommand{\nm}{{\it nautical miles}}
\newcommand{\ms}{${\rm m\,s^{-1}}$}
\newcommand{\mms}{${\rm mm\,s^{-1}}\,$}
\newcommand{\cms}{${\rm cm\,s^{-1}}\,$}
\newcommand{\kgm}{${\rm kg\,m^{-1}}\,$}
\newcommand{\kgmmm}{${\rm kg\,m^{-3}}\,$}
\newcommand{\dgr}{$^{\circ}$}
\newcommand{\mmn}{$\rm m^{-2}$}
\newcommand{\mn}{$\rm m^{-1}$}

\newcommand{\DP}[2]{\frac{\partial #1}{\partial #2}}
\newcommand{\DF}[2]{\frac{#1}{ #2}}

\def \p{\partial}
\def \d{\mathrm{d}}
\def \D{\mathrm{D}}


%% Code syntax highlighting

\usepackage{listings}
\usepackage{color}

% Subfigures/captions
\usepackage{caption}
\usepackage{subcaption}

% Footnotes in tables
\usepackage{tablefootnote}

\definecolor{mygreen}{rgb}{0,0.6,0}
\definecolor{mygray}{rgb}{0.5,0.5,0.5}
\definecolor{mymauve}{rgb}{0.58,0,0.82}
\definecolor{mylightgray}{rgb}{0.95,0.95,0.95}

\lstset{frame=single,
    % language=C++,
    aboveskip=3mm,
    belowskip=3mm,
    showstringspaces=false,
    columns=flexible,
    basicstyle=\ttfamily,
    numbers=none,
    numberstyle=\tiny\color{mygray},
    keywordstyle=\bfseries\color{blue!40!black},
    commentstyle=\color{gray},
    stringstyle=\color{mygreen},
    % otherkeywords={for,if,else},
    breaklines=true,
    breakatwhitespace=true,
    tabsize=4,
    escapeinside={\%*}{*)},
    keepspaces=true,
    captionpos=b,
    backgroundcolor=\color{mylightgray}
}

% Pandoc tightlist
\providecommand{\tightlist}{%
  \setlength{\itemsep}{0pt}\setlength{\parskip}{0pt}}


%%% Heading spacing fixes
%%% For some reason this doesn't work inside the class
\makeatletter
\renewcommand\section{\@startsection{section}{1}{\z@}%
                    {-4ex \@plus -1ex \@minus -.2ex}%
                    {1ex \@plus .2ex}%
                    {\normalfont\large\bfseries}}

\renewcommand\subsection{\@startsection{subsection}{2}{\z@}%
                       {-3ex \@plus -1ex \@minus -.2ex}%
                       {0.001ex \@plus .2ex}%
                       {\normalfont\normalsize\bfseries}}

\renewcommand\subsubsection{\@startsection{subsubsection}{3}{\z@}%
                          {-3ex \@plus -1ex \@minus -.2ex}%
                          {0.001ex \@plus .2ex}%
                          {\normalfont\normalsize\bfseries}}
\makeatother
