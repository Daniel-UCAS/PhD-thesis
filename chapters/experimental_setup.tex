\chapter{Developing an experimental setup for measuring the performance and
near-wake of cross-flow turbines at large laboratory scale}

The first objective was to acquire high-fidelity experimental data to get a
clear picture of what it is we were trying to predict. We were looking to
acquire both performance data (mechanical power, drag force) and velocity data
in the near-wake. To do this, it was necessary to upgrade the UNH tow tank's
linear motion, control, and data acquisition systems, along with the turbine
test bed instrumentation developed in \cite{Bachant2011MS}.

\section{Modifications to the UNH tow tank}

\begin{table}
\centering
\begin{tabular}{c|c|c}
Spec & Old system & Target \\ 
\hline
Maximum speed & 1.4 m/s  & 3.0 m/s \\ 
Maximum acceleration & 0.1 m/s$^2$ & 2.0 m/s$^2$ \\ 
Control system & Open loop velocity only & Closed loop position \\ 
On-board power & $4\times12$ V batteries & Continuous 120 and 220 VAC \\ 
\end{tabular}
\caption{Specifications summary for existing and upgraded tow tank systems.} 
\label{tab:tow-tank-specs}
\end{table}

\subsection{Linear guides}

The previous linear guide system consisted of a ``master'' guide constructed from
$4 \times 4$ inch fiberglass tubing, and a ``slave'' guide constructed from aluminum angle, on which plastic wheels rode. Over time, the fiberglass tubing had failed structurally and was covered with stainless steel bars fixed with double-sided tape. These bars shifted around considerably during towing and were a source of noise.

A new set of linear guides was designed from 1.25 inch diameter Thomson 440C
stainless steel linear shafts and super self-aligning linear bearings. The
existing carriage was modified to retrofit the linear bearings, and a series of
parts were designed to adapt the stainless shafts to the existing quasi-level
mounting surfaces, which helped keep cost down.

\subsection{Motion and control}

The tow tank's previous motion system consisted of a 10 horsepower AC induction
motor powered by a Yaskawa V7 variable frequency drive. The motor was coupled to
a speed reducing gearbox, on which a pulley was mounted to drive a 0.25 inch
diameter wire rope. It was seen in previous testing that this system had very
low acceleration ($\sim 0.1$ m/s$^2$), which severely reduced steady state
towing durations. The relatively low spring constant of the wire rope tow member
also gave the system a low natural frequency, which resonated due to cross-flow
turbines' cycling forcing. Furthermore, the system was only velocity-controlled,
and in an open-loop manner. This meant positioning was done manually, which took
a skilled operator, and reduced usable tank length further to allow for
coasting to a stop.

These issues were addressed by changing the motor to a permanent magnet servo
motor, sized to tow turbines with 1 m$^2$ frontal area up to 3 m/s, while
accelerating at 2 m/$^2$. The motor was powered by a Kollmorgen S700 servo
drive, controlled by an ACS NTM EtherCAT master controller, providing closed
loop position (and velocity) control. A series of emergency stop buttons were
installed to dramatically increase the safety of the system.

A steel-reinforced polyurethane timing belt was chosen as the new drive member.
The most robust timing belt profile---an ATL20---was chosen for maximum
stiffness per unit width, increasing the drive member spring constant roughly by a
factor of 7.

\subsection{Data acquisition and on-board accessories}


\section{Upgraded turbine test bed}


\subsection{Wake measurement system}


\subsection{Software}

Software was developed to automate the entire turbine testing process. Dubbed
\textit{TurbineDAQ}, the desktop application was written in Python due to its
reputation as a good ``glue'' language for systems integration. The graphical
user interface (GUI) was built using the PyQt bindings to the Qt framework.
Communication with the tow tank's motion controller, data acquisition system,
and ADV were integrated into a single application.