\chapter{Developing an experimental setup for measuring the performance and
near-wake of cross-flow turbines at large laboratory scale}

The first objective was to acquire high-fidelity experimental data to get a
clear picture of what it is we were trying to predict. We were looking to
acquire both performance data (mechanical power, drag force) and velocity data
in the near-wake. To do this, it was necessary to upgrade the UNH tow tank's
linear motion, control, and data acquisition systems, along with the turbine
test bed instrumentation developed in \cite{Bachant2011MS}.

\section{Modifications to the UNH tow tank}

\begin{table}
\centering
\begin{tabular}{c|c|c}
Spec & Old system & Target \\ 
\hline
Maximum speed & 1.4 m/s  & 3.0 m/s \\ 
Maximum acceleration & 0.1 m/s$^2$ & 2.0 m/s$^2$ \\ 
Control system & Open loop velocity only & Closed loop position \\ 
On-board power & $4\times12$ V batteries & Continuous 120 and 220 VAC \\ 
\end{tabular}
\caption{Specifications summary for existing and upgraded tow tank systems.} 
\label{tab:tow-tank-specs}
\end{table}

\subsection{Linear guides}

The previous linear guide system consisted of a ``master'' guide constructed from
$4 \times 4$ inch fiberglass tubing, and a ``slave'' guide constructed from aluminum angle, on which plastic wheels rode. Over time, the fiberglass tubing had failed structurally and was covered with stainless steel bars fixed with double-sided tape. These bars shifted around considerably during towing and were a source of noise.

A new set of linear guides was designed from 1.25 inch diameter Thomson 440C
stainless steel linear shafts and super self-aligning linear bearings. The
existing carriage was modified to retrofit the linear bearings, and a series of
parts were designed to adapt the stainless shafts to the existing quasi-level
mounting surfaces, which helped keep cost down.

\subsection{Motion and control}

The tow tank's previous motion system consisted of a 10 horsepower AC induction
motor powered by a Yaskawa V7 variable frequency drive. The motor was coupled to
a speed reducing gearbox, on which a pulley was mounted to drive a 0.25 inch
diameter wire rope. It was seen in previous testing that this system had very
low acceleration ($\sim 0.1$ m/s$^2$), which severely reduced steady state
towing durations. The relatively low spring constant of the wire rope tow member
also gave the system a low natural frequency, which resonated due to cross-flow
turbines' cycling forcing. Furthermore, the system was only velocity-controlled,
and in an open-loop manner. This meant positioning was done manually, which took
a skilled operator, and reduced usable tank length further to allow for
coasting to a stop.

These issues were addressed by changing the motor to a permanent magnet servo
motor, sized to tow turbines with 1 m$^2$ frontal area up to 3 m/s, while
accelerating at 2 m/s$^2$. The motor was powered by a Kollmorgen S700 servo
drive, controlled by an ACS NTM EtherCAT master controller, providing closed
loop position (and velocity) control. A series of emergency stop buttons were
installed to dramatically increase the safety of the system.

A steel-reinforced polyurethane timing belt was chosen as the new drive member.
The most robust timing belt profile---an ATL20---was chosen for maximum
stiffness per unit width, increasing the drive member spring constant roughly by a
factor of 7.

\subsection{Data acquisition and on-board accessories}


\section{Upgraded turbine test bed}


\subsection{Wake measurement system}


\subsection{Software}

Software was developed to automate the entire turbine testing process. Dubbed
\textit{TurbineDAQ}, the desktop application was written in Python due to its
reputation as a good ``glue'' language for systems integration. The graphical
user interface (GUI) was built using the PyQt bindings to the Qt framework.
Communication with the tow tank's motion controller, data acquisition system,
and ADV were integrated into a single application. This combined with the
ability to load and automatically execute test matrices in comma-separated value
(CSV) format allowed for experiments consisting of thousands of tows, where the
previous generation could only realistically achieve around 100.


\section{Turbine models}

Two physical turbine models were designed and built.

\subsection{UNH-RVAT}

A primary goal for the UNH-RVAT was geometric simplicity, for the sake of
replication in numerical models. The turbine was model constructed from straight
14 cm chord length NACA 0020 extrusions, used for both the blades and struts.
Blades were mounted at mid-chord and mid-span, having a length of 1 m and placed
at 1 m diameter.

% Below from JoT paper

The turbine model was designed to be geometrically simple, in the spirit of, but
not identical to the Sandia/DOE Reference Model 2 (RM2) CFT \cite{Neary2013,
    Barone2011}. Compared to the full-scale Sandia/DOE RM2, it would be
approximately a 1:6 scale model, albeit with higher solidity, or chord-to-radius
ratio. The turbine rotor is $H=1$ m tall, has a $D=1$ m diameter and was
constructed from three NACA 0020 section blades with constant $c=0.14$ m chord
length, mounted at half-chord and half-span with zero preset blade pitch. The
support struts are also NACA 0020 sections with 0.14 m chord length, and these
are fixed to a $0.095$ m diameter shaft. A sketch of the turbine rotor is shown
in Figure~\ref{fig-expsetup} and CAD models are available from
\cite{Bachant2014_CAD}.


\subsection{DOE/SNL RM2}

The turbine is to be a 1:6 scale model of the RM2 rotor. Turbine geometry is to
be scaled from the RM2 ``rev 0'' design report \cite{Barone2011}, with the
exception of the shaft diameter, which will be a scaled version of the SAFL RM2
shaft \cite{Hill2014}. The hub design is also similar to the SAFL model, which
may aid in comparison of the results, though this is not a top priority.
Geometric parameters are shown in Table~\ref{tab:turb_geom} and a drawing of the
turbine design is shown in Figure~\ref{fig:RM2-drawing}. The turbine model
components---blades, struts, shaft, and center hub sections---will be fabricated
from 6061-T6 aluminum, which will be hardcoat anodized per MIL-8625-A, type III,
class 2 specifications.

\begin{table}[ht]
    \centering
    \begin{tabular}{l|l|l}
        & Full-scale & Model (1:6) \\
        \hline 
        Diameter (m)   & 6.450 & 1.075 \\ 
        Height (m)     & 4.840 & 0.8067 \\ 
        Blade root chord (m) & 0.4000 & 0.06667 \\ 
        Blade tip chord (m)  & 0.2400 & 0.04000 \\ 
        Blade profile & NACA 0021 & NACA 0021 \\ 
        Blade mount & 1/2 chord & 1/2 chord \\ 
        Blade pitch (deg.) & 0.0 & 0.0 \\ 
        Strut profile & NACA 0021 & NACA 0021 \\ 
        Strut chord (m) & 0.3600 & 0.06000 \\ 
        Shaft diameter (m) & 0.2540 \cite{Beam2011} or 0.4160 \cite{Hill2014} & 0.06350\\ 
    \end{tabular}
    \caption{RM2 turbine geometric parameters.}
    \label{tab:turb_geom}
\end{table}

\begin{figure}[ht]
    \centering
%    \includegraphics[width=0.5\textwidth]{Figures/turbine}
    \caption{Illustration of the UNH RM2 scaled physical model.}
    \label{fig:RM2-drawing}
\end{figure}