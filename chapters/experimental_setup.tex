\chapter{Developing an experimental setup for measuring the performance and
near-wake of cross-flow turbines at large laboratory scale}

The first objective was to acquire high-fidelity experimental data to get a
clear picture of what it is we were trying to predict. We were looking to
acquire both performance data (mechanical power, drag force) and velocity data
in the near-wake. To do this, it was necessary to upgrade the UNH tow tank's
linear motion, control, and data acquisition systems, along with the turbine
test bed instrumentation developed in \cite{Bachant2011MS}.

\section{Experimental setup}


\section{Modifications to the UNH tow tank}

\begin{table}
\centering
\begin{tabular}{c|c|c}
Spec & Old system & Target \\ 
\hline
Maximum speed & 1.4 m/s  & 3.0 m/s \\ 
Maximum acceleration & 0.1 m/s$^2$ & 2.0 m/s$^2$ \\ 
Control system & Open loop velocity only & Closed loop position \\ 
On-board power & $4\times12$ V batteries & Continuous 120 and 220 VAC \\ 
\end{tabular}
\caption{Specifications summary for existing and upgraded tow tank systems.} 
\label{tab:tow-tank-specs}
\end{table}

\subsection{Linear guides}


\subsection{Motion control}


\section{Upgraded turbine test bed}