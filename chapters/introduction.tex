\chapter{Introduction}
\pagenumbering{arabic} % <- Do this command only for the first chapter
                                    
This research proposal is based on the following premises:

\begin{itemize}

	\item Cross-flow turbines (CFTs) are worth considering for some wind and marine
	hydrokinetic (MHK) energy applications.
	
	\item Engineers currently do not have adequate computationally affordable
	modeling tools to design these types of turbines. Because behavior cannot be
	predicted easily, devices cannot be optimized.
	
	\item Present methods for array design need at the very least to be
	recalibrated for CFTs. Higher fidelity models fail to capture the benefits of
	using cross-flow turbines instead of axial-flow turbines, rendering them
	somewhat useless.
	
	\item Actuator line models (ALMs) allow for reasonably accurate performance and
	detailed flow predictions in Navier--Stokes simulations, with lower
	computational cost compared with blade-resolved techniques, making them ideal
	for high fidelity array analysis.
 
\end{itemize}

Since there is currently no open-source ALM for cross-flow turbines, it is
proposed that one be developed for use with the OpenFOAM computational fluid
dynamics framework. This model is to be validated with extensive tow tank
experiments, performed at sufficiently high Reynolds number, using turbines with
significantly different geometric parameters.


\section{What are cross-flow turbines and why do they matter?}

A cross-flow turbine (CFT) is a device that converts a fluid's kinetic energy to
shaft work, which has its axis of rotation perpendicular to the flow. A familiar
example is the Darrieus vertical-axis wind turbine (VAWT) concept, investigated
thoroughly in the late 1970s through the early 1990s by groups including Sandia
National Laboratories (SNL) in the US and the National Research Council (NRC) in
Canada \cite{Para2002}. The CFT concept has been nearly abandoned for
large-scale onshore wind, but today is being applied or considered for floating
offshore wind \cite{Sandia2012, Paulsen2011}, smaller-scale wind
\cite{Giges2012}, and marine hydrokinetic (MHK) energy conversion, e.g., in
tidal and river flows.

There are good reasons why the three-bladed, horizontal-axis propeller-type
axial-flow turbine (AFT) concept has become the design of choice for large scale
onshore wind. Axial-flow turbines are easier to analyze (for higher power
coefficients $C_P = P/(1/2 \rho A U_\infty^3)$) since their operating principles
can be though of as an essentially steady flow over a foil before stall. AFTs
also have the benefit of research ``inertia''---a lot has been invested and a
lot of knowledge has accumulated already. As a result, the designs are quite
mature, for wind energy at least. In contrast, the CFT has been studied and
applied significantly less, though there have been cases where CFTs have
performed nearly equivalently well as AFTs. However, CFTs are harder to design,
since their blades are constantly changing their angles of attack throughout the
turbine's rotation, often undergoing dynamic stall as part of normal operation
\cite{Para2002}. Beside their unpredictability, the highly oscillatory blade
loading presents significant design challenges for avoiding fatigue---a main
cause for failure or premature retirement of the large Darrieus wind turbines.

Despite their shortcomings, CFTs still may be valuable in some cases. They are
simpler, omni-directional machines, which negates the need for yawing and
pitching mechanisms. In the wind field, CFTs show promise for projects where
space is limited, e.g., urban environments, as adjacent turbines sometimes
interact ``constructively'' to increase each other's power outputs
\cite{Li2010}---a trait not possessed by AFTs. A group from the California
Institute of Technology has shown that arrays of VAWTs can potentially provide
an order-of-magnitude increase in the power output per land area of a wind
turbine array, since the devices can be spaced more closely than HAWTs
\cite{Dabiri2011}. Field measurements have shown that their wakes recover more
quickly than those of AFTs, and this cannot be entirely attributed to higher
turbulence generation \cite{Kinzel2012}.

For MHK development, a field much less mature than wind power, the cross-flow
turbine concept is playing a major role thanks to its omni-directionality and
flexibility with respect to frontal area shape, allowing for more precisely
tuned fitment in channels with complex bathymetry or installed structures.
Chosen by the Ocean Renewable Power Company (ORPC) for their TidGen design, a
cross-flow turbine was the first grid-connected tidal energy device in the US,
which was installed in Cobscook Bay, ME \cite{ORPC2012}. Today, ORPC is working
on improving their turbine's performance, and planning for the installation of 4
more turbines \cite{Nelson2013}. As will be discussed in the following section,
present predictive capability is significantly less than adequate, for both
individual devices and arrays, which is a primary motivation for this research.


\section{The state of engineering tools for CFTs}

\subsection{For individual devices}

Presently, the most reliable predictor of turbine performance is physical
modeling (building prototypes or copying existing designs that have been
field-tested), so long as important dynamical scales (Reynolds number, mainly)
are sufficiently matched. It has been shown that performance becomes essentially
Reynolds number independent at an approximate blade chord Reynolds number $Re_c
\approx \lambda U_\infty c / \nu = O(10^5)$ \cite{Bravo2007}, which was
confirmed by experiments with a reference turbine in the UNH tow tank
\cite{Bachant2014}. However, experiments at this scale can be quite expensive.
For example, a turbine with a 1 m diameter and 10 cm chord would need to be
tested in a flow on the order of 1 m/s in water and 10 m/s in air. For a turbine
this large it is typically impractical to manufacture many prototypes to find
the optimal design, so numerical modeling is preferred. There exists a large
spectrum of numerical modeling techniques with widely varying computational cost
and fidelity, and sometimes only the most complex and computationally expensive
are trustworthy.

Momentum models are the simplest and cheapest, where the turbine blades are
discretized into blade elements, for which 2-D static lift and drag data are
tabulated. The relative velocity and angle of attack for a blade element are
calculated by seeking a balance between forces computed from the static foil
data and the rate of change of momentum of the fluid passing by the blade
element. Momentum methods break down for large streamwise forces, i.e.,
``induction factors''---common in water and high solidity turbines---and must be
corrected empirically. Despite their deficiencies, these models can do a
reasonable job for low solidity (ratio of blade planform area to swept area)
turbines when combined with corrections for dynamic loading, the most prominent
cause of which is dynamic stall \cite{Para2002}. A double multiple streamtube
(DMS) momentum model can compute a full turbine performance curve in seconds on
a modern desktop computer.

Vortex line methods are similar to momentum models, except blade element local
velocity is computed using potential flow theory, where lifting bodies are bound
vortex lines that shed vortex wake elements whose influences are combined via
the Biot--Savart law \cite{Strickland1979}. Sandia National Labs' CACTUS is an
example of a vortex method \cite{Murray2011}. CACTUS has been tested against
experimental data from large, low-solidity wind turbines (those for which
momentum models do well), but has been shown to fail for smaller turbines in
water, which are typically higher solidity \cite{Michelen2014}. Regarding
computing effort, vortex line methods can compute a full turbine performance
curve in minutes---slightly longer than the DMS method, but still quite fast.
The increase in accuracy of the flow field prediction and the robustness with
respect to high turbine loading justify the slightly higher expense of the
vortex line method.

A more sophisticated vortex model is the so-called panel method, where turbine
geometry can be specified arbitrarily as potential flow boundary elements,
negating the need for sectional foil coefficient tables. This is a significantly
more computationally expensive model. A single turbine operating point (not a
curve) may take hours on a conventional desktop PC. Furthermore, boundary layer
models are necessary to predict the occurrence and consequences of dynamic
stall \cite{Zanon2012}.

The most computationally expensive models solve the Navier--Stokes equations,
with turbulence modeled with Reynolds-averaging (RANS) or large eddy simulation
(LES)---the former being relatively less expensive, since LES directly solves a
larger portion of the energy spectrum of turbulence. If a body-fitted grid is
used, the actual turbine geometry is included as part of the computational
domain, and the mesh is generally refined next to the solid surfaces to resolve
the boundary layer, i.e., with cells adjacent to walls having a nondimensional
wall distance $y^+ \sim 1$. When only run in two dimensions, RANS methods are
affordable enough to be run on a single CPU, computing a single turbine
operating point in hours. However, 3-D effects are important enough that 2-D
simulations are not reliable, at least as predictors of absolute performance
\cite{Li2013}. 3-D simulations with a body-fitted grid are very expensive
(especially for LES), therefore are practically limited to high performance
computing (HPC) clusters. Even with the high computational expense, these models
are not perfect and results can deviate significantly from experimental
measurements, especially when using RANS models \cite{Li2013}.

Actuator line models (ALMs) are a mixture of the blade element and
Navier--Stokes methods. The turbine is not part of the mesh, but is represented
by lines that move through the flow, acting as momentum sinks, where the
resultant force is computed using 2-D foil data. Negating the need for a
body-fitted grid and resolving the boundary layer removes a significant amount
of computational effort, but the flow field is still computed more accurately
than with momentum or vortex methods, since nonlinear effects and turbulence are
included in the RANS or LES equations.


\subsection{For arrays}

Effective turbine array engineering directly depends on accurate prediction of
turbine wake generation, evolution and interaction, along with the impact of
various types of turbulent inflow on power production of each device. Like for
individual turbines, physical modeling is an option for predicting array
performance, though it becomes even more expensive to match relevant dynamical
scales. For this reason, turbine arrays are mainly designed using numerical
methods.

The contemporary industry standard method for predicting array performance
involves the superposition of prescribed wakes \cite{Stevens2014b}. Evolution
can be dependent on a single expansion coefficient chosen by the free stream
turbulence intensity \cite{Jensen1983, Choi2013} or computed by a solution of
the linearized RANS equations with an empirically derived constant eddy
viscosity closure \cite{Ainslie1988}. In light of the CFT's unique near-wake
dynamics, the valitidy of these models is questionable. At the very least they
would need to be recalibrated for CFTs, though their applicability is limited in
the near-wake of any turbine, meaning they are generally inappropriate for
closely-spaced arrays.
	
The next step up in complexity is the actuator disk method, where a constant
body force is added to the Navier--Stokes equations. This method can be
computationally cheap with RANS, or quite expensive and thorough with LES. The
ORPC turbine array is being laid out using the SNL-EFDC code, which uses a
constant uniform force applied to the RANS equations, where the turbine injects
turbulence kinetic energy and dissipation for the model's $k$--$\epsilon$
closure \cite{Nelson2013}. These models will most likely fail to predict the
fast wake recovery of CFTs, as they do not resolve the blade forces that create
the unique mean flow field in the near-wake \cite{Bachant2014_JOT}.
	
At present simulations with body-fitted grids are limited to one or two turbines
due to computational cost, which means they are impractical for full array
simulations. Thus the actuator line method, when combined with LES, is the most
complex model being used today. The ALM has the benefit of resolving unsteady
flow features created by periodic blade forcing and end effects, ultimately
producing the most accurate parameterization for turbine induced forces in
Navier--Stokes simulations. It has been shown in blind axial-flow turbine
modeling tests that ALM/LES methods fair better when predicting turbine induced
turbulence, and therefore will be more accurate at predicting flow within a
turbine array \cite{Krogstad2013}.


\section{Developing the cross-flow turbine actuator line model}

As described in the previous section, the actuator line model is valuable in
that it can be used to predict both the performance and wake of a turbine. It
can also be applied to high fidelity modeling of turbine arrays thanks to its
time-resolved nature. The ALM also eliminates the need for complex grids, which
is a significant simplification since mesh generation is arguably the largest
impediment to automation in CFD \cite{Slotnick2014}. Computational effort will
also be significantly reduced since the ALM does not need a rotating mesh, and
small cells near the turbine blade surface can be eliminated.

The proposed model will have some of the drawbacks of momentum and vortex
methods for predicting performance, namely the reliance on static foil data.
However, the flow field, and therefore relative velocity and blade angle of
attack will be predicted more accurately, and should therefore produce more
accurate predictions of turbine performance and wake characteristics.

\subsection{Experimental work}

In order to benchmark numerical models and further understand the flow physics
of the CFT wake, we have developed an automated turbine test bed. Note that this
work included a complete redesign of the UNH tow tank's motion, control, and
data acquisition systems. We have also developed a reference vertical-axis
turbine (UNH-RVAT) and collected a baseline performance and near-wake dataset,
which has been made publicly available \cite{Bachant2014_data}. From these
measurements we have identified that the unique mean velocity field in the
near-wake advects momentum and energy axially, which contributes significantly
to near-wake recovery \cite{Bachant2014_JOT}.

We have also preliminarily quantifed scaling ($Re$) effects to ensure results
will be relevant to full-scale, confirming that performance becomes Reynolds
number independent at $Re_c \approx 2 \times 10^5$, and near-wake
characteristics slightly lower \cite{Bachant2014}. Further analysis will be done
to analyze the effects of $Re$ on the relative importance of mean and turbulent
flow mechanisms governing near-wake dynamics, similar to that of
\cite{Bachant2014_JOT}, to assess the ability of scaled physical model arrays to
predict accurately the flow through full-scale arrays.

Regarding additional experimental work, an SNL/DOE reference model ``RM2''
cross-flow turbine will be built and an additional dataset will be collected.
These tests will help diagnose issues with CACTUS by also measuring the effects
of strut drag. They will also provide a validation case for the ALM using a
turbine with relatively low solidity---4\% versus the RVAT's 13\%.

\subsection{Numerical work with body-fitted grids}

To obtain a closer look at what is happening in the flow field around the RVAT,
3-D RANS simulations of the actual turbine geometry are being performed using
SNL's Red Mesa HPC cluster. If these simulations can correctly postdict the
blade loading and near-wake measurements, they can be used to ``interpolate''
the experimental results and provide more insight for validation of the ALM, or
to possibly diagnose and remedy some of the flaws within the CACTUS code.
Currently some test simulations have been run, showing promising results, and a
2-D slice of the model is being used to check for grid convergence.

\subsection{ALM mathematical model}

The actuator line model, originally developed by Sorensen and Shen
\cite{Sorensen2002}, treats turbine blades as lines of blade elements, for which
2-D foil lift and drag coefficients are tabulated. For any given blade element,
relative flow velocity and angle of attack are computed by adding the vectors
of relative blade motion and the local fluid velocity. The blade lift and drag
forces are calculated using

\begin{equation}
F_L = \frac{1}{2} \rho A_\mathrm{elem} C_L U_\mathrm{rel}^2,
\end{equation}
and
\begin{equation}
F_D = \frac{1}{2} \rho A_\mathrm{elem} C_D U_\mathrm{rel}^2,
\end{equation}
where $\rho$ is the fluid density, $A_\mathrm{elem}$ is the blade element
planform area (span $\times$ chord), $U_\mathrm{rel}$ is the local relative
velocity, and $C_L$ and $C_D$ are the sectional lift and drag coefficients,
chosen from a table via the local angle of attack and typically the Reynolds
number. The forces are then projected onto the rotor coordinate system to
calculate torque, overall drag, etc. Forces from the turbine shaft and blade
support struts will be computed in a similar way.

The use of static foil data necessitates corrections for various dynamic effects
induced by the the actuator lines rotating within the flow field. Dynamic stall
may be encountered when the blade angle of attack increases past a certain
threshold, and is characterized by an initial increase in lift as a vortex is
shed from the foil's leading edge, after which a drop in lift occurs as the
vortex is advected downstream. Dynamic stall has been shown to be a significant
positive contributor to performance in CFTs \cite{Para2002, Urbina2013}, so it
is important to choose an accurate model. Following common examples in the
literature, including CACTUS, we will start with the Leishmann--Beddoes model
due to its favorable performance in, for example, Scheurich and Brown
\cite{Scheurich2011}. The dynamic stall model will also likely benefit from the
low Mach number modification of Sheng et al. \cite{Sheng2008}.

Other dynamic loading considerations stem from the blade rotation and
acceleration---so-called pitching circulation, flow curvature, and added-mass
effects. We will start by using the correction methodology of CACTUS, which is
based on an analogy with a pitching flat plate, using the nondimensional pitch
rate of the blade element to correct normal and tangential force coefficients
\cite{Murray2011}. It is possible that these corrections will not produce
adequate results, but by using the Navier--Stokes equations to solve the flow
field we at least will have removed the uncertainty due to wake vortex
instability discussed in Michelen et al. \cite{Michelen2014}.

There is an important difference, however, in the way the body force is imparted
on the flow field. In the vortex method, singularities exist by design, but in a
Navier--Stokes simulation on a grid, a singularity of force caused by, e.g., a
blade element in a single cell, can cause numerical instability. We will adopt
the Gaussian projection method of Sorensen and Shen \cite{Sorensen2002}, which
is also used in the National Renewable Energy Laboratory's (NREL) Simulator for
Wind Farm Applications (SOWFA), a comprehensive library for ALM/LES wind farm
simulations in OpenFOAM \cite{Churchfield2013}.

If the proposed body force model is inadequate for postdicting our experimental
performance data, one possible strategy for improving accuracy is to generate
foil coefficient databases with more dimensions, e.g., blade pitch rate or local
turbulence levels. It may also be possible to modify the dynamic loading models
based on insight from the body-fitted grid simulations.

\subsection{ALM software implementation}

The cross-flow turbine actuator line model (CFT-ALM) will be written into an
extension library for the OpenFOAM free and open-source CFD library. OpenFOAM is
in general a package for solving partial differential equations using the finite
volume method. However, it contains many pre-built applications for meshing,
pre-processing, solving, and post-processing CFD simulations. The library
contains implementations of models for compressible and incompressible flow,
turbulence via RANS, LES, and detatched eddy simulation (DES), multiphase flows,
dynamic mesh handling, and more, making it a very powerful general purpose
simulation framework, capable of easily scaling to massively parallel high
performance computing clusters. OpenFOAM is also widely used throughout industry
and academia, making it a perfect library in which to integrate this model, with
respect to potential impact and adoption.

The code will be written in OpenFOAM's style, similar to NREL's SOWFA, but will
use OpenFOAM's \texttt{fvOptions} framework for adding source terms to equations
at runtime. This strategy will allow the CFT-ALM to be added to many of the
solvers included in OpenFOAM, meaning it can be readily used with RANS or LES,
multiphase models (e.g. for simulating the free surface in MHK installations),
and even with heat transfer. Furthermore, unlike SOWFA, custom solvers will not
need to be developed to include the CFT-ALM. It is also likely that the
extension library developed here will be expanded to include axial flow turbine
models similar to SOWFA.

Since all applications are run from a command line and all input data is text
based, automation and integration with other tools is relatively
straightforward. Future enhancements could include cooperation with software for
generating static foil data, e.g., XFOIL or other OpenFOAM solvers, turbine
controller models, structural analysis codes, and optimization tools, e.g.,
SNL's DAKOTA, for both individual turbines and array layouts.

\subsection{Dissemination}
  
One important objective is to distribute research products (code, data, CAD
files, etc.) openly, such that they are as useful and reproducible as possible.
It is also important that the code is set up to grow and evolve via
contributions from the community---one of the most significant advantages of the
open-source development model.

Unfortunately, a lot of research code is merely described in publications, but
never released for adoption and improvement by the community. For example,
Shamsoddin and Porte-Agel have written a cross-flow turbine actuator line model
for LES, but have not made their code available or described results regarding
performance prediction \cite{Shamsoddin2014}. Even CACTUS, despite being
open-source, is simply distributed as a static archive for download, so there is
no clear procedure for users to contribute to the code.

To help correct this issue, the results of this research will be open-source and
hosted on GitHub.com to facilitate the inclusion of improvements by other
developers through their fork/pull mechanism. We have already implemented a
sharing strategy for experimental results using GitHub to host code and
processed data, and figshare.com for raw data, CAD files, and snapshots of code
or simulation settings used in specific publications \cite{Bachant2014_data,
Bachant2014_CAD, Bachant2014_Vortexje, Bachant2014_OF-AS}. We will continue to
share results in this way to ensure maximum impact from reuse, remixing, etc.
Furthermore, the library developed here will be openly available through all
stages of development (in fact, it is already), which will hopefully prevent
other developers from unknowingly working on similar projects, and collaborating
instead.

\section{Summary}

It is argued that the research community and industry alike would benefit from
the development of an open-source actuator line model to represent cross-flow
turbines in Navier--Stokes simulations---written as an extension to OpenFOAM, in
particular. It is proposed that this method will use a combination of the
standard ALM used for axial-flow turbines, with dynamic corrections similar to
SNL's CACTUS vortex line model, and the typical Gaussian projection of body
force onto the flow field.

In order to understand and validate the model, detailed experimental work has
been undertaken with a simple reference model turbine. Thus far these
measurements have revealed important mechanisms in the CFT near-wake that help
contribute to its relatively fast wake recovery, which will need to be captured
by the ALM. Experiments have also shown the consequences of scaled physical
models, confirming that we are likely operating in a Reynolds-number independent
regime, therefore model validation can be assumed to be approximately true for
full-scale applications. Acquiring another validation dataset for a turbine
with significantly different geometric parameters is planned, ensuring broad
applicability of the model.

Ultimately, the proposed work will produce a versatile engineering tool, which
can be used for individual turbine design on a few computing cores using
Reynold-averaged Navier--Stokes turbulence models, as well as for state of the
art research and design for turbine arrays with large eddy simulation and high
performance computing.