\chapter{Experimental characterization of a low-solidity cross-flow turbine}

For the low solidity RM2, the performance was characterized at multiple Reynolds
numbers, with the near-wake at only one, for comparing to the RVAT baseline
data.


\section{Turbine model}

The RM2 CFT was initially conceptualized by Sandia National Labs for the US
Department of Energy to be a generic case for numerical modeling
\cite{Barone2011}, specifically Sandia's CACTUS vortex model. With the blade and
strut parameters specified, a turbine model was designed and machined from
6061-T6 aluminum. The hub design mimicked that of the smaller scale RM2 build by
the Saint Anthony Falls Laboratory (SAFL) \cite{Hill2014}, though the blade to
strut connections were more streamlined.


\section{Experimental test plan}


\section{Results}


\subsection{Performance and Reynolds number dependence}


\subsection{Near-wake characteristics}


\section{Conclusions}

