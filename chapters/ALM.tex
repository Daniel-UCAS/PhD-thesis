\chapter{Developing an actuator line model for cross-flow turbines}

After getting a sense for how two different turbines perform, and how they
affect the flow in which they are placed, the goal is to now develop a way to
predict these effects without resorting to blade-resolved Navier--Stokes CFD.

The ultimate goal of the actuator line model is to drive down the computational
costs of simulating turbines by negating the need for complicated meshing, and
the subsequent boundary layer resolution.

\section{Model inputs}

% Not multiple Re coefficient tables!

\section{Blade element discretization}

\section{Unsteady aerodynamics}

\section{Flow curvature corrections}

\section{Reynolds number corrections}
As seen in Chapter~\ref{chap:Re-dep}

\section{Effects on turbulence modeling}

Conventional blade element simulations use either momentum or vortex methods to
solve for the incident flow field, and these methods do not model the effects of
turbulence. With the actuator line model, there is the opportunity to improve
the physical realism by not only adding a source to the momentum equations, but
also to the turbulence model equations.

In this case, we seek to ``inject'' turbulence dependent on blade loading, both
for the $k$--$\epsilon$ RANS model and the
\todo[inline]{Pick LES model to work with.}

\section{Software implementation}

\section{Results}

\subsection{RANS}

\subsection{LES}

\section{Computational cost}
% Compare with the blade-resolved RANS