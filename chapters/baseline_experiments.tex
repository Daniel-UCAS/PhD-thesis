\chapter{Baseline experimental characterization two model cross-flow turbines}

Two turbines were designed and built with similar scaling, but disparate
solidities, to provide baseline datasets to guide and validate the numerical
models developed later.

For the higher solidity turbine---the UNH-RVAT---the performance and near-wake
were measured in an initial experiment. The Reynolds number dependence,
described in Chapter~\ref{chap:Re-dep}, was observed in a separate experiment.

For the low solidity RM2, the performance was characterized at multiple Reynolds
numbers, with the near-wake at only one, for comparing to the RVAT baseline
data.

\section{Turbine models}

\subsection{UNH-RVAT}

A primary goal for the UNH-RVAT was geometric simplicity, for the sake of
replication in numerical models. The turbine was model constructed from straight
14 cm chord length NACA 0020 extrusions, used for both the blades and struts.
Blades were mounted at mid-chord and mid-span, having a length of 1 m and placed
at 1 m diameter.

\subsection{DOE RM2}

The RM2 CFT was initially conceptualized by Sandia National Labs for the US
Department of Energy to be a generic case for numerical modeling
\cite{Barone2011}, specifically Sandia's CACTUS vortex model. With the blade and
strut parameters specified, a turbine model was designed and machined from
6061-T6 aluminum. The hub design mimicked that of the smaller scale RM2 build by
the Saint Anthony Falls Laboratory (SAFL) \cite{Hill2014}, though the blade to
strut connections were more streamlined.

\section{Experimental test plans}

\subsection{UNH-RVAT}

\subsection{DOE RM2}