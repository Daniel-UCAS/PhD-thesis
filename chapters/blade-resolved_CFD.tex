\chapter{On the use of high-fidelity computational fluid dynamics}

Computing power has reached the level where it is feasible to simulate a
cross-flow turbine by solving the 3-D Reynolds-averaged Navier--Stokes equations
on a boundary layer-resolving body-fitted grid. It is of interest to determine
the accuracy of such a simulation, since as previously discussed the flow
physics of CFTs are very complex.


\section{Turbulence modeling}

Two popular RANS models were used to model turbulence---the Spalart--Allmaras
(SA) and Menter's $k$--$\omega$ shear stress transport (SST) models. These are
both eddy-viscosity type models, where the SA solves one additional scalar
transport equation and the SST two.


\section{Verification}

To verify that the simulations had reached a grid-independent solution in both
space and time, a 2-D slice of the model was run at various grid spacings, for
both turbulence models.


\section{Comparing cost with physical modeling}


\section{Conclusions}

2-D blade-resolved RANS simulations have been shown to be a poor predictor of
turbine performance.

3-D blade-resolved RANS simulations have been shown to be a fair predictor of
turbine performance and wake characteristics, though the computational cost is
too expensive to be used for engineering work, especially considering the
uncertainty involved compared with physical model studies.