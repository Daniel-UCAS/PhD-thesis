\chapter{Conclusions}

To help meet the need for more high quality cross-flow turbine performance and
wake data---especially for higher solidity rotors---an automated turbine test
bed was developed as part of UNH's wave and tow tank, which increased the number
of possible tows per experiment by an order of magnitude while improving
repeatability. Two ``large laboratory scale'' turbines ($\sim 1$ m scale) were
designed, built, and tested: the high solidity UNH-RVAT, and the medium solidity
DOE/SNL RM2.

A baseline performance and near-wake measurement dataset was acquired for the
UNH-RVAT. A new method for assessing wake recovery was developed by rearranging
the streamwise mean momentum and mean kinetic energy equations to examine their
streamwise partial derivatives \cite{Bachant2015-JoT}. Weighted averages of
these terms were calculated from experimental data to assess the relative
balance of various transport mechanisms. For the UNH-RVAT, it was shown that the
mean vertical advection dominated in the near-wake region, which is caused by
the unique vorticity field generated by the bound and tip vortices.

Scale, or Reynolds number effects on the UNH-RVAT data were examined by
remeasuring performance and near-wake data at multiple tow speeds. It was shown
that the measurements became nearly $Re$-independent at a turbine diameter
Reynolds number $Re_D \sim 10^6$, which provides a good guideline for keeping
physical model tests relevant to full scale behavior.

A similar experimental campaign was undertaken for the RM2. Reynolds number
dependence showed a similar threshold, though the performance retained weak
linear $Re$-dependence at the highest speeds tested. This effect was also seen
when inspecting the maximum geometric torque coefficient computed from 2-D
static foil coefficient data produced by the XFOIL viscous panel code. Since the
UNH-RVAT has a higher solidity or chord-to-radius ratio, and therefore higher
virtual camber, the Reynolds number independence of its performance---estimated
by the geometric torque coefficient calculation---is more dramatic. The
near-wake of the RM2 overall showed lower levels of streamwise recovery, though
mean vertical advection was still dominant. The turbulence was generated more
symmetrically due to the rotors higher operational tip speed ratio, caused by
its lower solidity.

A blade-resolved RANS CFD model was evaluated for its ability to postdict the
UNH-RVAT baseline data, using both 2-D and 3-D configurations, and the
Spalart--Allmaras and $k$--$\omega$ SST turbulence models. As expected, the 2-D
models overestimated performance due to higher blockage and lack of end effects.
The lack of the vertical dimension also made them poor predictors of overall
wake dynamics, which precludes their use for analyzing arrays of turbines like
the UNH-RVAT, despite the computational feasibility. The 3-D models performed
better, with the Spalart--Allmaras model postdicting mean performance closest to
the experimental results. Both models did a good job resolving the qualitative
features of the near-wake's mean velocity field. Overall, 3-D blade-resolved
RANS presents a potentially less expensive, though less trustworthy alternative
to $Re$-independent physical model testing. However, the cost of 3-D
blade-resolved CFD really depends on availability of high performance computing
resources, much like physical model testing depends on the availability of
experimental facilities.

It is argued that the research community and industry alike would benefit from
the development of an open-source actuator line model to represent cross-flow
turbines in Navier--Stokes simulations---written as an extension to OpenFOAM, in
particular. It is proposed that this method will use a combination of the
standard ALM used for axial-flow turbines, with dynamic corrections similar to
SNL's CACTUS vortex line model, and the typical Gaussian projection of body
force onto the flow field.

In order to understand and validate the model, detailed experimental work has
been undertaken with a simple reference model turbine. Thus far these
measurements have revealed important mechanisms in the CFT near-wake that help
contribute to its relatively fast wake recovery, which will need to be captured
by the ALM. Experiments have also shown the consequences of scaled physical
models, confirming that we are likely operating in a Reynolds-number independent
regime, therefore model validation can be assumed to be approximately true for
full-scale applications. Acquiring another validation dataset for a turbine
with significantly different geometric parameters is planned, ensuring broad
applicability of the model.

Ultimately, the proposed work will produce a versatile engineering tool, which
can be used for individual turbine design on a few computing cores using
Reynold-averaged Navier--Stokes turbulence models, as well as for state of the
art research and design for turbine arrays with large eddy simulation and high
performance computing.

Ideas for improving the ALM include modifying ALE force coefficients based on
local turbulence and/or vorticity levels, to provide a more accurate
representation of foil behavior in turbulence flows.