\chapter{Conclusions}

To help meet the need for more high quality cross-flow turbine performance and
wake data---especially for higher solidity rotors---an automated turbine test
bed was developed as part of UNH's wave and tow tank, which increased the number
of possible tows per experiment by an order of magnitude while improving
repeatability. Two ``large laboratory scale'' turbines ($\sim 1$ m scale) were
designed, built, and tested: the high solidity UNH-RVAT, and the medium solidity
DOE/SNL RM2.

A baseline performance and near-wake measurement dataset was acquired for the
UNH-RVAT. A new method for assessing wake recovery was developed by rearranging
the streamwise mean momentum and mean kinetic energy equations to examine their
streamwise partial derivatives \cite{Bachant2015-JoT}. Weighted averages of
these terms were calculated from experimental data to assess the relative
balance of various transport mechanisms. For the UNH-RVAT, it was shown that the
mean vertical advection dominated in the near-wake region, which is caused by
the unique vorticity field generated by the bound and tip vortices.

Scale, or Reynolds number effects on the UNH-RVAT data were examined by
remeasuring performance and near-wake data at multiple tow speeds. It was shown
that the measurements became nearly $Re$-independent at a turbine diameter
Reynolds number $Re_D \sim 10^6$, which provides a good guideline for keeping
physical model tests relevant to full scale behavior.

A similar experimental campaign was undertaken for the RM2. Reynolds number
dependence showed a similar threshold, though the performance retained weak
linear $Re$-dependence at the highest speeds tested. This effect was also seen
when inspecting the maximum geometric torque coefficient computed from 2-D
static foil coefficient data produced by the XFOIL viscous panel code. Since the
UNH-RVAT has a higher solidity or chord-to-radius ratio, and therefore higher
virtual camber, the Reynolds number independence of its performance---estimated
by the geometric torque coefficient calculation---is more dramatic. The
near-wake of the RM2 overall showed lower levels of streamwise recovery, though
mean vertical advection was still dominant. The turbulence was generated more
symmetrically due to the rotors higher operational tip speed ratio, caused by
its lower solidity.

A blade-resolved RANS CFD model was evaluated for its ability to postdict the
UNH-RVAT baseline data, using both 2-D and 3-D configurations, and the
Spalart--Allmaras and $k$--$\omega$ SST turbulence models. As expected, the 2-D
models overestimated performance due to higher blockage and lack of end effects.
The lack of the vertical dimension made them unable to realize the strongest
contribution to wake recovery---mean vertical advection----which precludes their
use for analyzing arrays of low aspect ratio turbines, despite the computational
feasibility.

The 3-D blade-resolved RANS models performed better---the Spalart--Allmaras
model postdicting mean performance just 6\% below the experimental results,
where $k$--$\omega$ overpredicted by 30\%. Both models were able to resolve the
effects of blade tip vortex shedding in the near-wake's mean velocity field, and
therefore the vertical advection. Overall, 3-D blade-resolved RANS presents a
potentially less expensive alternative to $Re$-independent physical model
testing, though there is still some uncertainty with respect to turbulence model
selection. However, the cost of 3-D blade-resolved CFD really depends on
availability of high performance computing resources, much like physical model
testing depends on the availability of experimental facilities.

Motivated by the prospect of reducing the computational cost of 3-D CFD
simulations, an actuator line model was developed for cross-flow turbines and
implemented in a standard $k$--$\epsilon$ RANS model. The ALM was coupled with a
Leishman--Beddoes type semi-empirical dynamic stall model, end effects
correction, flow curvature correction, and an added mass model. Both the
UNH-RVAT and RM2 turbines were modeled with the RANS ALM. The performance curve
for the UNH-RVAT matched the experiments almost perfectly for tip speed ratios
below that of maximum power output, but was overpredicted above. For the RM2,
however, the shape of the $C_P$--$\lambda$ curve was ``stretched'' such that
$C_{P_{\max}}$ occurred at higher $\lambda$, though $C_{P_{\max}}$ was only
underpredicted by approximately 9\% compared with experiments. To address this
issue, it is recommended that a more rigorous validation be performed with
respect to the LB DS model time constants. It is hypothesized that tip speed
ratio dependent time constants may improve mean performance predictions.

Wake predictions with the RANS ALM matched some of the qualitative near-wake
flow field characteristics seen in experiments, e.g., the mean vertical
advection, and production of turbulence kinetic energy on the $+y$ or upwind
facing side of the rotor. Neither of these matched perfectly with experiments,
but did a much better job representing a CFT wake than a simple actuator disk,
with minimal additional computational effort.

The UNH-RVAT was also modeled using large eddy simulation with a typical
Smagorinsky subgrid-scale model. Inside the LES, the ALM's mean performance
coefficient predictions at $\lambda=\lambda_0=1.9$ dropped about five percentage
points compared with RANS. However, the LES was able to resolve some of the
important qualitative features of the near-wake's mean velocity field, i.e., the
apparent mean vortex pair created by the blade tip vortex shedding. On the other
hand, levels of turbulence were significantly lower than those measured in the
experiment, which is thought to be an effect of the subgrid-scale modeling
delaying vortex breakdown. Therefore, it is recommended that wake measurements
be taken further downstream, and deeper investigation of SGS modeling be
undertaken in order to recommend which LES model might be most effective for
modeling arrays of cross-flow turbines.

On the whole, the work described here helps engineers better select initial CFT
rotor concepts, along with which methods should be used to predict their
suitability. When full scale prototyping is not feasible, physical modeling at
$Re_D \sim 10^6$ can be effective. Both of these methods are expensive, so it
may be desirable to use 3-D blade-resolved RANS, despite also being expensive,
and its uncertainty with respect to turbulence modeling. The actuator line model
presents a good alternative when budgets are constrained, allowing turbines to
be simulated in a 3-D Navier--Stokes model with typical computing resources,
i.e., dropping the expense by two to four orders of magnitude versus
blade-resolved CFD for LES and RANS, respectively. For designing arrays of
turbines, the ALM at present is probably the best balance between cost and
accuracy, as its computational expense can be adjusted by the turbulence
modeling fidelity. Furthermore, the effects of, e.g., a free surface or
temperature/density stratification can easily be incorporated into ALM
simulations, opening up many opportunities for future investigation.

The products of this research---datasets, processing code, CAD files, simulation
case files, and the newly developed ALM software library turbinesFoam---have
been made freely and openly available. Besides improving transparency and
reproducibility, the open research paradigm also accelerates progress through
collaboration, allowing researchers to build on each other's work rather than
start anew. Working openly has already improved this research thanks to external
contributions, and it will hopefully improve others' as time goes on.
