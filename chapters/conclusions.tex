\chapter{Conclusions}

It is argued that the research community and industry alike would benefit from
the development of an open-source actuator line model to represent cross-flow
turbines in Navier--Stokes simulations---written as an extension to OpenFOAM, in
particular. It is proposed that this method will use a combination of the
standard ALM used for axial-flow turbines, with dynamic corrections similar to
SNL's CACTUS vortex line model, and the typical Gaussian projection of body
force onto the flow field.

In order to understand and validate the model, detailed experimental work has
been undertaken with a simple reference model turbine. Thus far these
measurements have revealed important mechanisms in the CFT near-wake that help
contribute to its relatively fast wake recovery, which will need to be captured
by the ALM. Experiments have also shown the consequences of scaled physical
models, confirming that we are likely operating in a Reynolds-number independent
regime, therefore model validation can be assumed to be approximately true for
full-scale applications. Acquiring another validation dataset for a turbine
with significantly different geometric parameters is planned, ensuring broad
applicability of the model.

Ultimately, the proposed work will produce a versatile engineering tool, which
can be used for individual turbine design on a few computing cores using
Reynold-averaged Navier--Stokes turbulence models, as well as for state of the
art research and design for turbine arrays with large eddy simulation and high
performance computing.

Ideas for improving the ALM include modifying ALE force coefficients based on
local turbulence and/or vorticity levels, to provide a more accurate
representation of foil behavior in turbulence flows.