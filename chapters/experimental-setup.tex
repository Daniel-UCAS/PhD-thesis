\chapter{Developing an experimental setup for measuring the performance and
near-wake of cross-flow turbines at large laboratory scale}

In 2011, a turbine test bed was developed for measuring the performance
(mechanical power and overall rotor drag or thrust) of large laboratory scale
($O(1)$ m\textsuperscript{2} frontal area) cross-flow turbines in the University
of New Hampshire (UNH) tow tank \cite{Bachant2011-MS}, a 36.6 m long, 3.66 m
wide, and 2.44 m deep facility that was capable of towing up to approximately
1.4 m/s. The turbine-specific instrumentation consisted of a mounting frame
built from NACA 0020 hydrofoil struts, a hydraulic disk brake for turbine
loading, an Interface T8 200 Nm capacity rotary torque transducer, and a 54
pulse per rev magnetic pickup for measuring shaft speed. The frame was mounted
to the carriage via streamwise-oriented linear bearings and was held in place by
a pair of Sentran ZB3 500 lbf load cells, to measure the overall streamwise
rotor drag.

Despite its usefulness in producing a small amount of data for two helical
cross-flow turbines~\cite{Bachant2015-RE}, the existing system had some issues
to be addressed:
\begin{itemize}
    \item No control over turbine shaft angular velocity. This made operation at
    tip speed ratio below peak torque impossible.
    
    \item Fully manual starting and load application. This limited resolution of
    the applied torque, and took considerable effort to perform experiments on
    the order of 100 tows, since a person had to ride on the carriage to adjust
    and apply the load torque.
    
    \item Open loop speed and manual position control of the tow carriage. This
    also took considerable effort to operate experiments, since the operator had
    to estimate braking distance to ensure the carriage did not hit the tank
    ends.
    
    \item Low carriage acceleration. The carriage acceleration was on the order
    of 0.2 m/s\textsuperscript{2}, which limited the steady state turbine
    operating duration to a few seconds.
    
    \item Low frequency resonance in the tow member. A long 0.25 inch diameter
    wire rope was used to tow the carriage, which resonated longitudinally with
    the significant variation of streamwise forces from the turbine.
\end{itemize}

In addition to the above issues, in order to meet the data acquisition goals, it
was necessary to measure turbine wake flows. These concerns required major
renovations, upgrades, and additions to the tow tank and turbine test bed
motion, control, and data acquisition systems. Furthermore, it was desirable to
automate the entire system to increase both data quality and quantity. These
changes were made possible thanks to an infrastructure grant from the US
Department of Energy (DOE).


\section{Modifications to the UNH tow tank}

\begin{table}
\centering
\begin{tabular}{c|c|c}
Spec & Old system & Target \\ 
\hline
Maximum speed & 1.4 m/s  & 3.0 m/s \\ 
Maximum acceleration & 0.1 m/s$^2$ & 2.0 m/s$^2$ \\ 
Control system & Open loop velocity only & Closed loop position \\ 
On-board power & $4\times12$ V batteries & Continuous 120 and 220 VAC \\ 
\end{tabular}
\caption{Specifications summary for existing and upgraded tow tank systems.} 
\label{tab:tow-tank-specs}
\end{table}

\subsection{Linear guides}

The previous linear guide system consisted of a ``master'' guide constructed
from $4 \times 4$ inch fiberglass tubing, and a ``slave'' guide constructed from
aluminum angle, on which plastic wheels rode. Over time, the fiberglass tubing
had failed structurally and was covered with stainless steel bars fixed with
double-sided tape. These bars shifted around considerably during towing and were
a source of noise.

A new set of linear guides was designed from 1.25 inch diameter Thomson 440C
stainless steel linear shafts and super self-aligning linear bearings. The
existing carriage was modified to retrofit the linear bearings, and a series of
parts were designed to adapt the stainless shafts to the existing quasi-level
mounting surfaces, which helped keep cost down.


\subsection{Motion and control}

The tow tank's previous motion system consisted of a 10 horsepower AC induction
motor powered by a Yaskawa V7 variable frequency drive. The motor was coupled to
a speed reducing gearbox, on which a pulley was mounted to drive a 0.25 inch
diameter wire rope. It was seen in previous testing that this system had very
low acceleration ($\sim 0.1$ m/s$^2$), which severely reduced steady state
towing durations. The relatively low spring constant of the wire rope tow member
also gave the system a low natural frequency, which resonated due to cross-flow
turbines' cyclic forcing. Furthermore, the system was only velocity-controlled,
and in an open-loop manner. This meant positioning was done manually, which took
a skilled operator, and reduced usable tank length further to allow for coasting
to a stop.

These issues were addressed by changing the motor to a permanent magnet servo
motor, sized to tow turbines with 1 m$^2$ frontal area up to 3 m/s, while
accelerating at 2 m/s$^2$. The motor was powered by a Kollmorgen S700 servo
drive, controlled by an ACS NTM EtherCAT master controller, providing closed
loop position (and velocity) control. A series of emergency stop buttons were
installed to dramatically increase the safety of the system.

A steel-reinforced polyurethane timing belt was chosen as the new drive member.
The most robust timing belt profile---an ATL20---was chosen for maximum
stiffness per unit width, increasing the drive member spring constant roughly by a
factor of 7.


%%% From RM2 test plan %%%

Experiments will be performed in the UNH tow/wave tank, a 36 m long facility
with a 3.66 m wide by 2.44 m deep cross-section, capable of tow speeds up to 3
m/s\footnote{Note that though 3 m/s is the technical limit, the practical limit
    for achieving substantial tow durations is approximately 2 m/s.}, pictured in
Figure~\ref{fig:tow-tank}. The turbine will be mounted in a frame built from
NACA 0020 struts, attached to the tow carriage by four linear bearings, which
transfer all streamwise force to a pair of S-beam load cells. The turbine shaft
RPM will be controlled by a servo motor system, which allows prescription of the
turbine tip speed ratio. The load torque will be measured by an inline rotary
torque transducer and a load cell mounted at a fixed distance from the servo
motor, providing a redundant measurement. Turbine shaft angle will be measured
using the servo drive's emulated encoder output, set to $10^5$ counts per
turbine shaft revolution. Carriage speed, and therefore inflow velocity will be
measured using a linear encoder with 10 $\mu$m resolution. All of these
performance-related quantities will be sampled as 2 kHz, while the tow tank's
motion controller will provide redundant measurements of the carriage speed and
turbine angular velocity sampled at 1 kHz. Turbine wake measurements at 1
turbine diameter downstream will be measured with a Nortek Vectrino+ acoustic
Doppler velocimeter, sampling at 200 Hz. A list of the sensors to be used in the
experiment is shown in Table~\ref{tab:sensors}, instrumentation in
Table~\ref{tab:instrumentation}, and a drawing of the experimental setup is
shown in Figure~\ref{fig:exp-setup}.


\begin{figure}[ht!]
    \centering 
%    \includegraphics[clip,trim=0 0.4in 0 0.37in, 
%    width=0.49\textwidth]{Figures/tow_tank_length} 
%    \includegraphics[clip,trim=0.67in 0 0 0, 
%    width=0.49\textwidth]{Figures/test_bed_photo} 
    \caption{Photos of the UNH towing tank and turbine test bed.} 
    \label{fig:tow-tank}
\end{figure}


\begin{table}[ht]
    \centering
    \begin{tabular}{c|c|c|c}
        Measured quantity & Device type & Mfg. \& model & Nominal accuracy \\
        \hline 
        Carriage position & Linear encoder & Renishaw LM15 & 10 $\mu$m/pulse \cite{RenishawLM15}\\
        Turbine angle & Servo encoder output & Kollmorgen AKD & 10$^5$ pulse/rev \cite{KollmorgenAKD}\\
        Turbine torque & Rotary transducer & Interface T8-200 & $\pm$0.5 Nm \cite{InterfaceT8}\\ 
        Turbine torque (2) & Load cell (\& arm) & Sentran ZB3-200 & $\pm$0.2 Nm \cite{SentranZB}\\
        Drag force, left & Load cell & Sentran ZB3-500 & $\pm$0.6 N \cite{SentranZB}\\
        Drag force, right & Load cell & Sentran ZB3-500 & $\pm$0.6 N \cite{SentranZB}\\
        Fluid velocity & ADV & Nortek Vectrino+ & $\pm$0.5\% $\pm$1 mm/s \cite{NortekVectrino}\\
    \end{tabular}
    \caption{Details of the sensors to be used for the experiment. Note that ``(2)''
        denotes a secondary redundant measurement. ``Turbine torque (2)'' nominal
        accuracy estimated by combining load cell accuracy and arm machining tolerances
        ($\pm 1 \times 10^{-4}$ m) as root-sum-square.} \label{tab:sensors}
\end{table}

\begin{table}[ht]
    \centering
    \begin{tabular}{c|c|c}
        Measured quantity & Device type & Mfg. \& model \\
        \hline 
        Carriage position & Differential counter & NI 9411 \\
        Carriage velocity (2) & Motion controller & ACS NTM \\
        Turbine angle & Differential counter & NI 9411 \\
        Turbine RPM (2) & Motion controller & ACS NTM \\
        Turbine torque & Analog voltage input & NI 9405 \\ 
        Turbine torque (2) & Analog bridge input & NI 9237 \\
        Drag force, left & Analog bridge input & NI 9237 \\
        Drag force, right & Analog bridge input & NI 9237 \\
    \end{tabular}
    \caption{Details of the instrumentation to be used for the experiment. Note that
        ``(2)'' denotes a secondary redundant measurement.}
    \label{tab:instrumentation}
\end{table}

\begin{figure}[ht]
    \centering
%    \includegraphics[clip,trim=0.01in 0 0 0, width=0.95\textwidth]{Figures/tank_cross_section}
    \caption{Illustration of the experimental setup.}
    \label{fig:exp-setup}
\end{figure}


\begin{figure}[ht]
    \centering
    %\begin{subfigure}[t]{\textwidth}
    %    \centering
    %    \includegraphics[width=0.7\textwidth]{figures/exp-setup-photo}
    %    \caption{}
    %    \label{fig:exp-setup-photo}
    %\end{subfigure}
    
    %\begin{subfigure}[t]{\textwidth}
    %    \centering
    %    \includegraphics[width=0.7\textwidth]{figures/exp_setup_drawing}
    %    \captiton{}
    %    \label{fig:exp-setup-dwg}
    %\end{subfigure}
    
    \caption{Experimental setup photo (a) and drawing (b): turbine test bed
        installed in the UNH tow tank.}
    
    \label{fig:exp-setup}
\end{figure}


\subsection{Calibrations}

Before collecting data, traceable calibration certificates will be obtained for
the Interface T8-200 torque transducer and NI 9405 and NI 9237 modules. The
torque transducer calibration will be used directly in data processing, while
the drag measurement load cells will be calibrated using an additional Sentran
ZB S-beam load cell and indicator, a package which will also have its own
calibration certificate. This load cell will be mounted to a fixture that allows
varying load on each drag load cell---while the linear bearings are
installed---by a lead screw.  The redundant torque measurement load cell/arm
system will be calibrated in a similar fashion using the Sentran load cell and
indicator, by attaching it to a fixture with a distance from the axis of
rotation known to within approximately 0.005 inches.



%%% End section from RM2


\subsection{Data acquisition and on-board accessories}

The previous generation tow tank data acquisition (DAQ) system was based around
an on-board PC, powered by a set of four 12 V automotive batteries. This was
done to avoid the complexity of running power out to the carriage
\cite{Darnell1996}. The DAQ PC was accessed via Windows Remote Desktop to
control any DAQ applications. The PC that sent the control signal to the
inverter drive was a separate machine, which meant users had to work with at
least two interfaces to specify DAQ and motion parameters. This also made it
difficult to synchronize motion with data acquisition, e.g., triggering data
collection at a certain location.

Requirements for upgrades were derived from the goal of fully automating both
motion and data acquisition. It was also determined that the UNH ME department's
high frame rate particle image velocimetry (HFR-PIV) system would be used on the
carriage at some point, which included laser power supplies and a laser chiller
that could not be powered by the previous generation's isolated battery/inverter
system.

\todo[inline]{Calculate how long the PIV system could run on batteries}


\section{Upgraded turbine test bed}

For this work, the turbine test bed was kept mostly intact, but modified for
fully-automated operation. To reduce low frequency resonance in the frame caused
by turbine side forces, and help redistribute some of the streamwise force from
turbines towed at higher speeds, two pairs of steel guy wires were added. These
solutions were chosen based on a finite element analysis (FEA) of the turbine
mounting frame, which showed more improvement regarding stiffening in the
desired directions compared to simply adding 45 degree flat bar braces in the
corner joints. To ensure drag from the outer guy wires was included in the
overall streamwise force measurement, an additional set of linear bearings was
added to the carriage for their connection.


\subsection{Turbine loading and speed control}

In order to control turbine shaft angular velocity, a servo motor and gearhead were added with a custom retrofit mounting bracket.
Two zero-backlash R+W curved jaw couplings were added above and below the rotary torque transducer.
An additional torque measurement system was added by mounting the servo/gearhead assembly to a slewing ring bearing, and holding its mounting bracket in place by a load cell attached at a fixed distance by a 16 inch long arm. 


%%% From RVAT-Re-dep

Experiments were performed in a turbine test bed specifically designed for
cross-flow turbines. The test bed was integrated as part of the University of
New Hampshire (UNH) tow tank, which is a 36 m-long facility with a 3.66 m-wide
and 2.44 m-deep cross-section. The turbine model used in this study was the UNH
Reference Vertical Axis Turbine (RVAT), which was designed to be a generic case
for numerical model testing, similar to the Sandia National Labs/U.S. Department
of Energy Reference Model 2 (RM2) River Turbine \cite{Neary2014}, but with a
higher solidity or blade chord-to-radius ratio.

The turbine was mounted in a frame constructed from NACA 0020 sections, shown in
Figure~\ref{fig:exp-setup}. The turbine shaft ran up through the water surface,
coupled to a Kollmorgen AKM permanent magnet servo motor (Kollmorgen, Radford,
VA, USA) with a 20:1 gearbox, providing precise control over shaft angular
velocity. This servo was controlled by the tow tank's main motion controller for
high synchronization with the carriage motion, thereby giving precise
measurement and control of the tip speed ratio. An Interface T8 200 Nm capacity
rotary torque transducer (Interface, Scottsdale, AZ, USA) was installed inline
between the servo and the turbine, and the servo was also mounted on a slewing
ring bearing, which allowed a redundant measurement of torque via an arm and
load cell used to counteract the turbine moment. The frame was mounted to the
carriage via linear guides, such that the total streamwise drag force was
transferred to a pair of Sentran ZB3 500 pound-force capacity S-beam load cells
(Sentran, Santa Ana, CA, USA), providing the rotor drag measurements, after a
separately measured tare drag was subtracted in post-processing. Similarly, a
tare torque was measured by rotating the turbine shaft in air. Turbine angular
and tow carriage linear position were measured using quadrature encoder signals,
with $10^5$ counts-per-rev for the turbine and 10 ${\mu}$m resolution for the
carriage position. These~signals, along with the torque and drag signals, were
sampled at 2 kHz.

Wake velocity was measured using a Nortek Vectrino+ acoustic Doppler velocimeter
(ADV) (Nortek AS, Rud, Norway), which has an approximately 6~mm diameter
sampling volume and sampled at 200 Hz. The probe was mounted on an automated
positioning system, also controlled by the tow tank's main motion controller.
The ADV and data acquisition systems' sampling times were synchronized by
triggering the start of data acquisition via a pulse sent from the motion
controller. Additional details of the turbine and experimental setup are
described in \cite{Bachant2015-JoT}.

%%% End from RVAT-Re-dep


\subsection{Wake measurement system}

In order to characterize turbine wakes, a Nortek Vectrino+ acoustic Doppler
velocimeter (ADV) was purchased as part of the upgraded tow tank
instrumentation. An ADV is capable of measuring three components of velocity at
a single point in space, and the Vectrino+ can sample at 200 Hz. This system is
considered desirable compared with hot wire or hot film anemometry as there are
no calibrations, and the sensor element is significantly more robust. Spatial
resolution is typically lower---on the order of 1 cm \cite{NortekVectrino}---but
this is still small compared with the typical length scale of a turbine model.
ADV is also preferable to laser Doppler velocimetry (LDV) in this case since the
tow carriage is a high vibration environment, which would make LDV alignment a
challenge.

A $y$--$z$ axis positioning system was designed for the Vectrino probe. This
system consisted of two Velmex BiSlide linear stages---the $y$-axis driven by
belt and the $z$ by ball screw. Both drive systems were powered by stepper
motors with approximately 0.001 inch resolution. These motors were driven by an
ACS UDMlc EtherCAT drive, connected to the tow tank's main motion controller for
integrated synchronous motion.


\subsection{Software}

Software was developed to automate the entire turbine testing process. Dubbed
\textit{TurbineDAQ}, the desktop application was written in Python due to its
reputation as a good ``glue'' language for systems integration. The graphical
user interface (GUI) was built using the PyQt bindings to the Qt framework.
Communication with the tow tank's motion controller, data acquisition system,
and ADV were integrated into a single application. This combined with the
ability to load and automatically execute test matrices in comma-separated value
(CSV) format allowed for experiments consisting of thousands of tows, where the
previous generation could only realistically achieve around 100.


\subsection{Tare drag and torque compensation} 

The drag and torque measurement systems were set up in such a way that raw
measurements for drag would include all submerged gear and torque would include
all friction below the transducer along with the turbine shaft torque. To
compensate, tare torque and drag runs were to be performed to measure the shaft
bearing friction torque and turbine mounting frame drag, respectively. These
data will be similar to the turbine performance data, omitting torque
measurements for the tare drag runs and vice versa. Tare drag runs will be
performed for each tow speed in the experiment, for which the mean value is used
in data processing. Tare torque runs will be performed by rotating the turbine
shaft (without blades) in air at constant angular velocity for a specified
duration, over the range of angular velocities used throughout the experiment.
Tare torque will then be fit with a linear regression versus shaft angular
velocity, and added to the measured turbine torque in post-processing.


\subsection{Synchronization of instrumentation subsystems}

The three data acquisition instrumentation subsystems---motion controller, NI
DAQ (performance measurements), and Vectrino+ (wake velocity
measurements)---were set to begin sampling at precisely the same time each run,
after being triggered by a TTL pulse created by the motion controller. This
strategy retains synchronization for all performance signal samples (tow speed,
torque, drag, angular velocity), ensuring precise calculation of, e.g., power
coefficient. Since there is also synchronization of the initial sample from each
three subsystems, correlation of events in the performance and wake signals is
also possible.


\section{Determining tank settling time}

Each experiment, sample tows were done to determine the amount of time taken
between runs such that the tank has settled adequately, i.e., background
turbulence and any large scale mean flows have been dissipated. This was
assessed by towing the turbine, then allowing the Vectrino to continue recording
velocity data, monitoring the mean and standard deviation of the signals. The
settling times were then in the experiment configuration---one value for each
tow speed, to be used by \textit{TurbineDAQ} as wait times between automated
runs.


\section{Uncertainty of experimental measurements}


%%% Below from RVAT-Re-dep paper


Uncertainty was considered from a combination of systematic and random errors.
The random error was inferred from the sample standard deviation (on a
per-revolution basis) and the systematic error from the sensor calibrations or
datasheets. Combining both sources of error, along with their propagation into
quantities derived from multiple measurements, followed the procedures outlined
in Coleman and Steele \cite{ColemanSteele}, described below.

An expanded uncertainty interval with 95\% confidence was computed for
$C_P$, $C_D$, and mean wake velocities:
\begin{equation}
U_{95} = t_{95} u_c,
\end{equation}
where $t_{95}$ is the value from the Student $t$-distribution for a 95\%
confidence interval and $u_c$ is the combined standard uncertainty. Combined
standard uncertainty for a given quantity $X$ is calculated~by:
\begin{equation}
u_X^2 = s_{\bar{X}}^2 + b_X^2,
\end{equation}
where $s_{\bar{X}}$ is the sample standard deviation of the mean per turbine
revolution, and $b_X$ is the systematic uncertainty, computed by:
\begin{equation}
b_{X}^2 = \sum_{i=1}^J \left( \frac{\partial X}{\partial x_i} \right)^2
b_{x_i}^2,
\end{equation}
where $x_i$ is a primitive quantity used to calculate $X$ (e.g., $T$, $\omega$,
and $U_\infty$ for calculating $C_P$), and $b_{x_i}$ is the primitive quantity's
systematic uncertainty, estimated as half the value listed on the sensor
manufacturer's documentation.

Selecting $t_{95}$ requires an estimate for degrees of freedom $\nu_X$, which
was obtained using the Welch--Satterthwaite formula:
\begin{equation}
\nu_X = \frac{\left(s_X^2 + \sum_{k=1}^M b_k^2 \right)^2} {s_X^4/\nu_{s_X} +
    \sum_{k=1}^M b_k^4/\nu_{b_k}},
\end{equation}
where $\nu_{s_X}$ is the number of degrees of freedom associated with $s_X$ and
$\nu_{b_k}$ is the number of degrees of freedom associated with $b_k$.
$\nu_{s_X}$ is assumed to be $(N-1)$, where $N$ is the number of independent
samples (or turbine revolutions). $\nu_{b_k}$ was estimated as:
\begin{equation}
\nu_{b_k} = \frac{1}{2} \left( \frac{\Delta b_k}{b_k} \right)^{-2},
\end{equation}
where the quantity in parentheses is the relative uncertainty of $b_k$, assumed
to be 0.25.


%%% End from RVAT-Re-dep paper


\section{Blockage}

Putting a turbine in a confined environment such as a towing tank will force
flow through the turbine at higher velocity compared with a free case, where
streamlines are allowed to diverge. Consequently, higher levels of blockage will
lead to increased turbine performance, and a shift in optimal operating
parameters, i.e., tip speed ratio. In order to for experiments to be relevant to
others in the literature, blockage effects must either be corrected for, or the
blockage ratio should be kept to a typical value seen in other studies.


\section{Turbine models}

Two physical turbine models were designed and built. The first was intended to
be a geometrically simple---i.e. symmetrical foil profiles, square frontal area,
rectangular blade planform---high-solidity turbine. This turbine was constructed
from materials donated by Lucid Energy Technologies, LLP, and dubbed the
``Reference Vertical-Axis Turbine'' or UNH-RVAT.

The second turbine was designed and built as part of a measurement task for
Sandia National Laboratories (SNL), in collaboration with the US Department of
Energy (DOE). The so-called ``Reference Model 2'' (RM2) was developed by SNL to
be a standard cross-flow turbine for which modelers could validate their
predictions \cite{Neary2014}. The RM2 was designed using Sandia's CACTUS vortex
line code \cite{Barone2011}, and its low solidity made it a nice complement to
the UNH-RVAT for testing the robustness of numerical models to varying solidity.


\subsection{UNH-RVAT}

A primary goal for the UNH-RVAT was geometric simplicity, for the sake of
replication in numerical models. The turbine was model constructed from straight
14 cm chord length NACA 0020 extrusions, used for both the blades and struts.
Blades were mounted at mid-chord and mid-span, having a length of 1 m and placed
at 1 m diameter.

% Below from JoT paper

The turbine model was designed to be geometrically simple, in the spirit of, but
not identical to the Sandia/DOE Reference Model 2 (RM2) CFT \cite{Neary2013,
    Barone2011}. Compared to the full-scale Sandia/DOE RM2, it would be
approximately a 1:6 scale model, albeit with higher solidity, or chord-to-radius
ratio. The turbine rotor is $H=1$ m tall, has a $D=1$ m diameter and was
constructed from three NACA 0020 section blades with constant $c=0.14$ m chord
length, mounted at half-chord and half-span with zero preset blade pitch. The
support struts are also NACA 0020 sections with 0.14 m chord length, and these
are fixed to a $0.095$ m diameter shaft. A sketch of the turbine rotor is shown
in Figure~\ref{fig:exp-setup} and CAD models are available from
\cite{Bachant2014-RVAT-CAD}.

% Below from RVAT-Re-dep paper

The UNH-RVAT turbine has three blades made from NACA 0020 profiles with a 0.14 m
chord. The~blades are mounted at mid-chord w.r.t. the turbine axis, and the
turbine has a height (blade span) of 1 m and a diameter of 1 m; \textit{cf}.
Figure~\ref{fig:turbine}. The blockage ratio produced by the rotor's frontal
area is 11\%, which we decided not to correct for, as reliable methods are not
yet agreed upon for CFTs \cite{Cavagnaro2014}. As such, blockage should be taken
into account when using these data for model validation, \emph{i.e.}, the
experimental domain should be mimicked. The rotor has a relatively high solidity
$Nc/(\pi D) = 0.13$ and a large chord-to-radius ratio $c/R = 0.28$. A CAD model
of the turbine is available from \cite{Bachant2014-RVAT-CAD}.

\begin{figure}[ht]
    \centering
    
    %\includegraphics[width=0.4\textwidth]{figures/turbine}
    
    \caption{University of New Hampshire Reference Vertical Axis Turbine (UNH-RVAT)
        turbine model. Turbine blades and struts made from NACA 0020 profiles with
        0.14~m chord. Note that the upper and lower mounting flanges have been excluded,
        as these were included in the tare drag measurements.}
    
    \label{fig:turbine}
\end{figure}


\subsection{DOE/SNL RM2}

The turbine was designed as a 1:6 scale model of the RM2 rotor described in the
RM2 ``rev 0'' design report \cite{Barone2011}, with the exception of the shaft
diameter, which was scaled from the SAFL RM2 shaft \cite{Hill2014}. The hub
design is also similar to the SAFL model, which may aid in comparison of the
results, though this is not a top priority. Geometric parameters are shown in
Table~\ref{tab:turb-geom} and a drawing of the turbine design is shown in
Figure~\ref{fig:RM2-drawing}. The turbine model components---blades, struts,
shaft, and center hub sections---will be fabricated from 6061-T6 aluminum, which
will be hardcoat anodized per MIL-8625-A, type III, class 2 specifications.

\begin{table}[ht]
    \centering
    \begin{tabular}{l|l|l}
        & Full-scale & Model (1:6) \\
        \hline 
        Diameter (m)   & 6.450 & 1.075 \\ 
        Height (m)     & 4.840 & 0.8067 \\ 
        Blade root chord (m) & 0.4000 & 0.06667 \\ 
        Blade tip chord (m)  & 0.2400 & 0.04000 \\ 
        Blade profile & NACA 0021 & NACA 0021 \\ 
        Blade mount & 1/2 chord & 1/2 chord \\ 
        Blade pitch (deg.) & 0.0 & 0.0 \\ 
        Strut profile & NACA 0021 & NACA 0021 \\ 
        Strut chord (m) & 0.3600 & 0.06000 \\ 
        Shaft diameter (m) & 0.2540 \cite{Beam2011} or 0.4160 \cite{Hill2014} & 0.06350\\ 
    \end{tabular}
    \caption{RM2 turbine geometric parameters.}
    \label{tab:turb-geom}
\end{table}

\begin{figure}[ht]
    \centering
%    \includegraphics[width=0.5\textwidth]{Figures/turbine}
    \caption{Illustration of the UNH RM2 scaled physical model.}
    \label{fig:RM2-drawing}
\end{figure}


\section{Summary and conclusions}

An upgraded test bed for large laboratory scale ($O(1)$ m) cross-flow turbines
was developed for the UNH tow tank. To acquire adequate amounts of data and to
achieve significant lengths of steady state operation with a limited tank
length, the tow tank's linear motion, control, and data acquisition systems had
to be redesigned, rebuilt, and upgraded to allow fully automated operation,
along with higher carriage speed and acceleration.

The upgraded test bed increased the number of tows possible per experiment by
essentially an order of magnitude.
