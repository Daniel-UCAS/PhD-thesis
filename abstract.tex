\begin{Abstractpage}

\setlength{\baselineskip}{1.5\baselineskip}
{

Cross-flow (often vertical-axis) turbines (CFTs), despite being thoroughly
investigated and subsequently abandoned for large scale wind energy, are seeing
renewed interest for smaller scale wind turbine arrays, offshore wind, and
marine hydrokinetic (MHK) energy applications. Though they are similar to the
large scale Darrieus wind turbines, today's CFT rotors are often designed with
higher solidity, or blade chord-to-radius ratios, which makes their behavior
more difficult to predict with numerical models. Furthermore, most experimental
datasets used for numerical model validation were acquired with low solidity
rotors.

An experimental campaign was undertaken to produce high quality open datasets
for the performance and near-wake flow dynamics of CFTs. An automated
experimental setup was developed using the University of New Hampshire's towing
tank. The tank's linear motion, control, and data acquisition systems were
redesigned and rebuilt to facilitate automated cross-flow turbine testing at
large laboratory (on the order of 1 meter) scale.

Two turbines were designed and built---one high solidity (dubbed the UNH
Reference Vertical-Axis Turbine or UNH-RVAT) and one medium-to-low solidity,
which was a scaled model of the US Department of Energy and Sandia National
Labs' Reference Model 2 (RM2) cross-flow MHK turbine. A baseline performance and
near-wake dataset was acquired for the UNH-RVAT, which revealed that the
relatively fast wake recovery observed in vertical-axis wind turbine arrays,
thought to be caused by increased turbulence compared with horizontal-axis wind
turbine wakes \cite{Kinzel2012}, could be attributed to the mean vertical
advection of momentum and energy, caused by the unique interaction of vorticity
shed from the blade tips.

The Reynolds number dependence of the UNH-RVAT was investigated by varying
turbine tow speeds, indicating that the baseline data had essentially achieved a
Reynolds number independent state at a turbine diameter Reynolds number $Re_D
\sim 10^6$ or chord based Reynolds number $Re_c \sim 10^5$. A similar study was
undertaken for the RM2, with similar results. An additional dataset was acquired
for the RM2 to investigate the effects of blade support strut drag on overall
performance, which showed that these effects can be quite significant---on the
order of percentage points of the power coefficient---especially for lower
solidity rotors, which operate at higher tip speed ratio. The wake of the RM2
also showed the significance of mean vertical advection on wake recovery, though
the lower solidity made these effects weaker than for the UNH-RVAT.

Blade-resolved Reynolds-averaged Navier--Stokes (RANS) computational fluid
dynamics (CFD) simulations were performed to assess their ability to model
performance and near-wake of the UNH-RVAT baseline case at optimal tip speed
ratio. In agreement with previous studies, the 2-D simulations were a poor
predictor of both the performance and near-wake. 3-D simulations faired much
better, but the choice of an appropriate turbulence model remains uncertain.
Furthermore, 3-D blade-resolved RANS modeling is computationally expensive,
requiring high performance computing (HPC), which may preclude its use for array
analysis.

Finally, an actuator line model (ALM) was developed to attempt to drive down the
cost of 3-D CFD simulations of cross-flow turbines, since previously, the ALM
had only been investigated for a very low Reyolds number 2-D CFT. Despite
retaining some of the disadvantages of the lower fidelity blade element momentum
and vortex methods, the ALM, when coupled with dynamic stall, flow curvature,
added mass, and end effects models, was able to predict the performance of
cross-flow turbines reasonably well. Near-wake predictions were able to match
some of the important qualitative flow features, which warrants further
validation farther downstream and with multiple turbines. Ultimately, the ALM
provides an attractive alternative to blade-resolved CFD, with computational
savings of two to four orders of magnitude for large eddy simulation and RANS,
respectively.

}


\end{Abstractpage}
