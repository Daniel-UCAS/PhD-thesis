%%%%%%%%%%%%%%%%%%%%%%%%%%%%%%%%%%%%%%%%%%%%%%%%%%%%%%%%%%%%%%%%%%%%%%%%%

%
% DEFINITIONS AND COMMANDS
%
%%%%%%%%%%%%%%%%%%%%%%%%%%%%%%%%%%%%%%%%%%%%%%%%%%%%%%%%%%%%%%%%%%%%%%%%%
%
% MARGINS:
%
%\textwidth 6.5in \oddsidemargin 0in \evensidemargin 0in \textheight
%8.5in \topmargin -3.5in

%
% LOAD PACKAGES:
%
\usepackage{amsmath}
\usepackage{amssymb}
\usepackage{amsthm}
%\usepackage{doublespace}    % use this package if compiling from UNIX (and comment setspace)
\usepackage{setspace}       % use this package if compiling from Windows using MikTex or LiveTex and comment doublespace
\usepackage{epsfig}
\usepackage{unh_thesis}
%\usepackage{harvard}
\usepackage{Times}

%%My own packages
\usepackage[T1]{fontenc}
\usepackage[left=1.55in,top=1.1in,right=1.1in,bottom=1in]{geometry}    %HP laserjet 3600
%\usepackage[colorlinks=true, pdfstartview=FitV, linkcolor=black, citecolor=black, urlcolor=black]{hyperref}
\usepackage{graphicx}
\usepackage[round]{natbib}
% packages added by Gopal
\usepackage{lscape}
\usepackage{multirow}
\usepackage{supertabular}
\usepackage{rotating}
%\usepackage{pstricks}
\usepackage{longtable}
\usepackage{threeparttable}
\usepackage{mathtools}
\usepackage{relsize}
%\usepackage{caption}
%\DeclareMathSizes{12}{20}{14}{10}  % For size 12 text
%\DeclareMathSizes{11}{19}{13}{9}   % For size 11 text
\DeclareMathSizes{10}{18}{12}{8}   % For size 10 text
% This package and new command were added by Gopal to use in the creating long tables, tables spanning multiple pages. %The issue  was that pre-defining the parboxes for a long table automatically justified the text. This command keeps %the box sizes and aligns the text to the left of the box. so instead of "p" for defining a parbox in a table an "x" is used which is \raggedright (means flushes to left)
\usepackage{array}
\newcolumntype{x}[1]{%
>{\raggedright\hspace{0pt}}p{#1}}%

%My own added commands
    \newcommand{\ee}{\end{equation}}
    \newcommand{\be}{\begin{equation}}
    \newcommand{\bi}{\begin{itemize}}
    \newcommand{\ei}{\end{itemize}}
    \newcommand{\bc}{\begin{center}}
    \newcommand{\ec}{\end{center}}
    \newcommand{\mc}{\multicolumn}

    \newcommand{\nm}{{\it nautical miles}}
    \newcommand{\ms}{${\rm m\,s^{-1}}$}
    \newcommand{\mms}{${\rm mm\,s^{-1}}\,$}
    \newcommand{\cms}{${\rm cm\,s^{-1}}\,$}
    \newcommand{\kgm}{${\rm kg\,m^{-1}}\,$}
    \newcommand{\kgmmm}{${\rm kg\,m^{-3}}\,$}
    \newcommand{\dgr}{$^{\circ}$}
    \newcommand{\mmn}{$\rm m^{-2}$}
    \newcommand{\mn}{$\rm m^{-1}$}

    \newcommand{\DP}[2]{\frac{\partial #1}{\partial #2}}
    \newcommand{\DF}[2]{\frac{#1}{ #2}}
%Commands related to figures
%    \setlength{\abovecaptionskip}{-1mm}
%    \renewcommand\floatpagefraction{.9}
%    \renewcommand\topfraction{.9}
%    \renewcommand\bottomfraction{.9}
%    \renewcommand\textfraction{.1}
%    \setcounter{totalnumber}{50}
%    \setcounter{topnumber}{50}
%    \setcounter{bottomnumber}{50}
%%%

\pagestyle{plain} % Puts page number on the bottom over the whole thesis


% BIBLIOGRAPHY STYLE:
%
%\bibliographystyle{myunsrt}
%\bibliographystyle{unsrt} % this was in Gagik's
%\bibliographystyle{apsrev}
%\bibliographystyle{harvard}
\bibliographystyle{abbrvnat}
%\bibliographystyle{abbrv}
