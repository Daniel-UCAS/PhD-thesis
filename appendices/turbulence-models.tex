\chapter{Turbulence models}\label{chap:turbulence-models}
\doublespace

In this chapter the three eddy viscosity RANS and one LES turbulence models used
throughout this work are detailed. Note that since all of these were applied via
OpenFOAM, their exact implementation can be interrogated by looking at the
source code, which is freely available.

The exact incompressible RANS equations are written as
\begin{equation}
    \frac{\partial U_i}{\partial t}
    + U_j \frac{\partial U_i}{\partial x_j}
    =
    -\frac{1}{\rho}\frac{\partial P}{\partial x_i}
    - \frac{\partial}{\partial x_j} \overline{u_i^\prime u_j^\prime}
    + \nu \frac{\partial^2 U_i}{\partial x_j \partial x_j}
\end{equation}

The RANS equations with the eddy viscosity formulation are modified with the
relation $-\overline{u_i^\prime u_j^\prime} = 2 \nu_t S_{ij}$, where $\nu_t$ is
the turbulence viscosity and $S_{ij}$ is the mean strain rate tensor. The RANS
equations with the eddy viscosity approximation then become
\begin{equation}
    \frac{\partial U_i}{\partial t}
    + U_j \frac{\partial U_i}{\partial x_j}
    =
    -\frac{1}{\rho}\frac{\partial P}{\partial x_i}
    + (\nu + \nu_t) \frac{\partial^2 U_i}{\partial x_j \partial x_j}.
\end{equation}


\section{Spalart--Allmaras}

The Spalart--Allmaras turbulence model is based on the transport of a single
scalar $\tilde{\nu}$ \cite{Spalart1992}, which is designed to behave linearly
near the wall. The transport equation for $\tilde{\nu}$ is defined as

% Eq. 9 from Spalart1992 (or 1994)
\begin{equation}
    \frac{\mathrm{D} \tilde{\nu}}{\mathrm{D} t}
    = c_{b1} \tilde{S} \tilde{\nu}
    + \frac{1}{\sigma}
    \left[
    \nabla \cdot \left( (\nu + \tilde{\nu}) \nabla \tilde{\nu} \right)
    + c_{b2} (\nabla \tilde{\nu})^2
    \right]
    - c_{w1} f_w \left[ \frac{\tilde{\nu}}{d} \right]^2,
    \label{eq:SA-nutilde}
\end{equation}
with
% Stilda from OpenFOAM
\begin{equation}
    \tilde{S} = f_{v3}\sqrt{2} | \mathrm{skew} \nabla U |
    + f_{v2} \frac{\tilde{\nu}}{\kappa^2 d^2},
\end{equation}
where $\mathrm{skew} \nabla U$ is the skew-symmetric component of the velocity
gradient tensor.

Auxiliary relations are defined by:
%fv2 and fv3 from OpenFOAM (no Ashford correction)
\begin{equation}
    f_{v2} = 1 - \frac{\chi}{1 + \chi f_{v1}}, \, \, \,
    f_{v3} = 1,
\end{equation}
% From OpenFOAM: Cw1_(Cb1_/sqr(kappa_) + (1.0 + Cb2_)/sigmaNut_)
\begin{equation}
    c_{w1} = c_{b1}/\kappa^2 + \frac{1 + c_{b2}}{\sigma},
\end{equation}
\begin{equation}
    f_w(r) = g \left[ \frac{1 + c_{w3}^6}{g^6 + c_{w3}^6} \right]^{1/6},
\end{equation}
\begin{equation}
    g = r + c_{w2} (r^6 - r),
\end{equation}
and
\begin{equation}
    f = \frac{\tilde{\nu}}{\tilde{S} \kappa^2 d^2},
\end{equation}
where $d$ is the distance to the wall. The SA turbulent viscosity is calculated
as $\nu_t = f_{v1} \tilde{\nu}$. Typical coefficients are
\begin{equation}
    c_{b1} = 0.1355, \, \,
    c_{b2} = 0.622, \, \,
    c_{w2} = 0.3, \, \,
    c_{w3} = 2.0, \, \,
    c_{v1} = 7.1, \, \,
    c_{v2} = 5.0, \, \,
    \sigma = 0.66666, \, \,
    \kappa = 0.41.
\end{equation}


\section{$k$--$\omega$ SST}

Menter's $k$--$\omega$ SST is defined in \cite{Menter2001}. The transport
equation for the turbulence kinetic energy $k$ is

\begin{equation}
    \rho \frac{\partial k}{\partial t}
    + \rho U_j \frac{\partial k}{\partial x_j}
    = \tilde{P}_k - \rho \beta^* \omega k
    + \frac{\partial}{\partial x_j}
    \left(
    \Gamma_k \frac{\partial k}{\partial x_j}
    \right),
    \label{eq:komegasst-k}
\end{equation}
and for the specific dissipation $\omega$:
\begin{equation}
    \rho \frac{\partial \omega}{\partial t}
    + \rho U_j \frac{\partial \omega}{\partial x_j}
    = \frac{\gamma}{\nu_t} P_k - \rho \beta \omega^2
    + \frac{\partial}{\partial x_j}
    \left(
    \Gamma_\omega \frac{\partial \omega}{\partial x_j}
    \right)
    + 2 \rho (1 - F_1) \sigma_{\omega 2}
    \frac{1}{\omega} \frac{\partial k}{\partial x_j}
    \frac{\partial \omega}{\partial x_j}.
\end{equation}

Auxiliary relations include:
\begin{equation}
    \Gamma_k = \mu + \frac{\mu_t}{\sigma_k}, \, \, \,
    \Gamma_\omega = \mu + \frac{\mu_t}{\sigma_\omega}, \, \, \,
    P_k = \tau_{ij} \frac{\partial U_i}{\partial x_j}, \, \, \,
    \tilde{P}_k = \min(P_k;c_l \epsilon), \, \, \,
    \mu_t = \rho \frac{a_1 k}{\max(a_1 \omega; S \cdot F_2)}.
\end{equation}

Coefficients $\phi$ are blended with $\phi = F_1 \phi_1 + (1 - F_1) \phi_2$,
where the 1 and 2 subscripts denote values from the $k$--$\omega$ and
$k$--$\epsilon$ models, respectively:
\begin{equation}
    F_1 = \tanh (\arg_1^4),
\end{equation}
\begin{equation}
    \arg_1 = \min
    \left(
    \max
    \left(
    \frac{\sqrt{k}}{\beta^* \omega y};
    \frac{500 \nu}{y^2 \omega}
    \right);
    \frac{4 \rho \sigma_{\omega2} k}{CD_{k \omega} y^2}
    \right),
\end{equation}
\begin{equation}
    CD_{k \omega} = \max
    \left(
    2 \rho \sigma_{\omega 2} \frac{1}{\omega}
    \frac{\partial k}{\partial x_j}
    \frac{\partial \omega}{\partial x_j};
    10^{-10}
    \right).
\end{equation}

OpenFOAM's implementation of the SST model features updated coefficients from
\cite{Menter2003}, with the consistent production terms from \cite{Menter2001}.
OpenFOAM also uses coefficients in terms of $\alpha = 1/\sigma$ to apply
blending functions. Default coefficients are
\begin{equation}
\begin{split}
&   \alpha_{k1} = 0.85, \, \, \,
    \alpha_{k2} = 1.0, \, \, \,
    \alpha_{\omega1} = 0.5, \, \, \,
    \alpha_{\omega2} = 0.856, \, \, \,
    \beta_1 = 0.075, \, \, \,
    \beta_2 = 0.0828, \, \, \,
    \beta^* = 0.09, \, \, \, \\
&   \gamma_1 = 5/9, \, \, \,
    \gamma_2 = 0.44, \, \, \,
    a_1 = 0.31, \, \, \,
    b_1 = 1.0, \, \, \,
    c_1 = 10.0.
\end{split}
\end{equation}


\section{$k$--$\epsilon$}

The standard $k$--$\epsilon$ model is defined in \cite{Wilcox1994}. The equation
for the turbulence kinetic energy is
\begin{equation}
    \rho \frac{\partial k}{\partial t}
    + \rho U_j \frac{\partial k}{\partial x_j}
    = \tau_{ij} \frac{\partial U_i}{\partial x_j}
    - \rho \epsilon
    + \frac{\partial}{\partial x_j}
    \left[
    (\mu + \mu_t/\sigma_k) \frac{\partial k}{\partial x_j}
    \right].
    \label{eq:kepsilon-k}
\end{equation}
and for the dissipation $\epsilon$:
\begin{equation}
    \rho \frac{\partial \epsilon}{\partial t}
    + \rho U_j \frac{\partial \epsilon}{\partial x_j}
    = C_{\epsilon 1} \frac{\epsilon}{k} \tau_{ij}
    \frac{\partial U_i}{\partial x_j}
    - C_{\epsilon 2} \rho \frac{\epsilon^2}{k}
    + \frac{\partial}{\partial x_j}
    \left[
    (\mu + \mu_t/\sigma_k) \frac{\partial \epsilon}{\partial x_j}
    \right].
    \label{eq:kepsilon-epsilon}
\end{equation}

The eddy-viscosity is defined as
\begin{equation}
    \nu_t = C_\mu k^2 / \epsilon,
    \label{eq:kepsilon-nut}
\end{equation}
and typical closure coefficients are
\begin{equation}
    C_{\epsilon 1} = 1.44, \, \,
    C_{\epsilon 2} = 1.92, \, \,
    C_\mu = 0.09, \, \,
    \sigma_k = 1.0, \, \,
    \sigma_\epsilon = 1.3.
\end{equation}


\section{Smagorinsky LES model}

Large eddy simulation solves for the filtered velocity field $\tilde{u}$. The
filtered Navier--Stokes equations are
\begin{equation}
    \frac{\partial \tilde{u}_i}{\partial t}
    + \tilde{u}_j \frac{\partial \tilde{u}_i}{\partial x_j}
    = - \frac{1}{\rho} \frac{\partial \tilde{p}}{\partial x_i}
    + \nu \frac{\partial^2 \tilde{u}_i}{\partial x_j \partial x_j}
    - \frac{\partial \tau_{ij}}{\partial x_j}.
    \label{eq:filtered-ns}
\end{equation}

Smagorinsky~\cite{Smagorinsky1963} proposed an eddy viscosity type relation for
the computing the subgrid scale (SGS) stress tensor $\tau_{ij}$:
% From Wilcox 1994 Eq 8.32
\begin{equation}
    \tau_{ij} = 2 \mu_t S_{ij}, \, \, \,
    S_{ij} = \frac{1}{2}
    \left(
    \frac{\partial \tilde{u}_i}{\partial u_j}
    + \frac{\partial \tilde{u}_j}{\partial u_i}
    \right).
\end{equation}
The eddy viscosity is then
\begin{equation}
    \nu_t = (C_S \Delta)^2 \sqrt{\tilde{S}_{ij} \tilde{S}_{ij}},
\end{equation}
where $\Delta$ is the filter width (e.g., the cube root of the local cell
volume) and $C_S$ is the Smagorinsky coefficient, which is not universal, i.e.,
will vary for different flow conditions. Note that OpenFOAM's implementation of
the Smagorinsky model is based on two coefficients, which by default are
\begin{equation}
    C_k = 0.094, \, \, \,
    C_\epsilon = 1.048.
\end{equation}
These are related to $C_S$ by
% http://blog.sina.com.cn/s/blog_6d9c27ab0100u9ez.html
% http://www.cfd-online.com/Forums/openfoam-solving/65838-smagorinsky-model-details.html
\begin{equation}
    C_S =
    \left(
        C_k \left(C_k / C_\epsilon \right)^{1/2}
    \right)^{1/2}.
\end{equation}
